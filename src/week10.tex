\section{Einstein convention}

Recall \(v\in \mathbb{R}^n\), \(v=(v^1,\ldots,v^n)\implies v=\sum_{i=1}^n v^i e_i\).
\begin{enumerate}[(1)]
    \item The summation is taken over \(i=1,2,\ldots,n\);
    \item \(v=\sum_{i=1}^n v^ie_i=\sum_{j=1}^nv^je_j\), expression of \(v\) is
        independent of choice of summation indices \(i\) or \(j\).
    \item One ``\(i\)'' appears as upper index, another ``\(i\)'' appears as lower
        index.
\end{enumerate}

Einstein convention: Whenever there is pair of same letter as upper and a lower
index, then the expression is summation of the index letter from 1 to \(n\),
\(n\) is usually the dimension of vector space / manifold / etc.

\begin{example}
\begin{itemize}
    \item \(W=V\cdot A\), \(A=(A_i^j)_{n\times n}\) matrix, \(v,w\in \mathbb{R}^n\),
        then \[
            W^j=v^i A_i^j=\sum_{i}v^iA_i^j
        .\] 
    \item \[
        g^{ij}\omega_{jk}=\sum_{j=1}^n g^{ij}\omega_{jk}.
    .\] Where \(g^{ij}\) is the \((i,j)\)-entry of inverse matrix \(g^{-1}\) of
    \(g\). \ie\ \[
        g^{ij}g_{jk}=\delta^i_k,
        \quad g^{ij}g_{ij}=\sum_{i=1}^n \sum_{j=1}^{n}g^{ij}g_{ij}
        =\sum_{i=1}^{n}\delta^i_i=n
    .\] 

    We will use Einstein convention from now on.
\end{itemize}
\end{example}

\section{Therema Egregium (Gauss)}

\underline{Goal:} The Gaussian curvature \(K\) depends only on the 1st fundamental
form.

Previously, we have seen once we know local parametrization \(\vphi(x^1,x^2)\) of
a surface \(S\), then \(I,\II\) can be computed and the Gaussian curvature is \[
    K=\frac{\det \II}{\det I}=\frac{eg-f^2}{EG-F^2}
.\] Now, we want to follow the same procedure as we study the Frenet formula of a
curve and try to understand the motion of equation of frame \(\vphi_1,\vphi_2,N\),
where \(\vphi_i=\pdv{\vphi}{x^i}, N=\frac{\vphi_1\times \vphi_2}{|\vphi_1\times 
\vphi_2|}\).

Fix a local parametrization \[
    \vphi\colon U \longrightarrow S\subset \mathbb{R}^3,
    \ (x^1,x^2)\longmapsto \vphi(x^1,x^2)
.\] Then
\begin{align*}
    I&=g_{ij}\dd{x^i}\dd{x^j},\quad g_{ij}=\left<\vphi_i,\vphi_j\right> \\
    \II&=h_{ij}\dd{x^i}\dd{x^j},\quad h_{ij}
.\end{align*}
We shall study the differential equation of \(\{\vphi_i,N\}\) up to 2nd order.
\begin{align*}
    \vphi_{11}&=\Gamma_{11}^1\vphi_1+\Gamma_{11}^1\vphi_2+h_{11}N\\
    \vphi_{12}&=\Gamma_{12}^1\vphi_1+\Gamma_{12}^1\vphi_2+h_{12}N\\
    \vphi_{21}&=\Gamma_{21}^1\vphi_1+\Gamma_{21}^1\vphi_2+h_{21}N\\
    \vphi_{22}&=\Gamma_{22}^1\vphi_1+\Gamma_{22}^1\vphi_2+h_{22}N\\
.\end{align*}
Weingarten equation, \(A=(a_i^j)\)
\[
    \begin{bmatrix}
        N_1 \\ N_2
    \end{bmatrix}=A\begin{bmatrix}
        \vphi_1 \\ \vphi_2
    \end{bmatrix}
.\] Then \[
    a_i^j=-h_{ik}g^{kj}
.\] Lets write above equations as
\begin{equation}\label{eq:motion}
    \begin{cases}
        \vphi_{ij}=\Gamma_{ij}^k\vphi_k+h_{ij}N \\
        N_i=a_i^j=\vphi_j
    \end{cases}
    \implies \text{The only unknown are }\Gamma_{ij}^k
.\end{equation}
Moreover, \[
    \vphi_{ij}=\vphi_{ji}\implies \Gamma_{ij}^k=\Gamma_{ji}^k
.\] \[
    \left<\vphi_{ij},\vphi_p\right> =\Gamma_{ij}^k \left<\vphi_k,\vphi_p\right> 
    =\Gamma_{ij}^kg_{kl}
.\] So
\begin{gather*}
    \pdv{g_{ip}}{x^j}-\left<\vphi_i,\vphi_{pj}\right> 
    =\left<\vphi_{ij},\vphi_p\right> =\Gamma_{ij}^kg_{kp} \\
    \pdv{g_{jp}}{x^i}-\left<\vphi_j,\vphi_{pi}\right> 
    =\left<\vphi_{ji},\vphi_p\right> =\Gamma_{ji}^kg_{kp} \\
    \pdv{g_{ij}}{x^p}=\left<\vphi_{ip},\vphi_j\right>
    +\left<\vphi_i,\vphi_{jp}\right> 
\end{gather*}
Hence \[
    2\Gamma_{ij}^k g_{kp}=\pdv{g_{jp}}{x^i}+\pdv{g_{ip}}{x^j}-\pdv{g_{ij}}{x_{p}}
.\] \[
    \implies 2\Gamma_{ij}^kg_{kp}g^{pq}
    =(\pdv{g_{jp}}{x^i}+\pdv{g_{ip}}{x^j}-\pdv{g_{ij}}{x_{p}})g^{pq}
.\] We get \[
    \Gamma_{ij}^k=\frac{1}{2}(\pdv{g_{jp}}{x^i}+\pdv{g_{ip}}{x^j}g^{pk}
    -\pdv{g_{ij}}{x_{p}})
.\] 

\begin{remark}
    We multiply \(g^{pk}\) form right because of the choice of row vector. In
    modern convention of the column vector, \[
        \Gamma_{ij}^k=\frac{1}{2}g^{kp}(\pdv{g_{jp}}{x^i}+\pdv{g_{ip}}{x^j}
        -\pdv{g_{ij}}{x_{p}})
    .\] 
\end{remark}
\begin{definition}
    \(\Gamma_{ij}^k\) is called the Christoffel symbols. They're uniquely determined
    by the 1st fundamental form.
\end{definition}

Next we derive the Gauss equation, we'll see the Gaussian curvature can be
expressed only in 1st fundamental form.

So far, we have obtained \[
    \vphi_{ij}=\Gamma_{ij}^k\vphi_k+h_{ij}N
    \quad\&\quad
    N_p=a_p^q\vphi_q
.\] Then
\begin{align*}
    \vphi_{ijp}&= \partial_p \Gamma_{ij}^k\vphi_k+\Gamma_{ij}^k\vphi_{kp}+
    \partial_ph_{ij} N+h_{ij}N_p \\
    &= \partial_p\Gamma_{ij}^k\vphi_k+\Gamma_{ij}^k(\Gamma_{kp}^q+h_{kp}N)
    +\partial_p h_{ij} N+h_{ij}a_p^q \vphi_q, \\
    \vphi_{ipj}
    &= \partial_j\Gamma_{ip}^k\vphi_k+\Gamma_{ip}^k(\Gamma_{kj}^q+h_{kj}N)
    +\partial_j h_{ip} N+h_{ip}a_j^q \vphi_q
.\end{align*}
The derivative is taken in \(\mathbb{R}^3\), so the two expression should be
the same, we got 
\begin{align*}
    \vphi_{ijp}-\vphi_{ipj}
    =&(\partial_p\Gamma_{ij}^k-\partial_j\Gamma_{ip}^k)\vphi_k+(\Gamma_{ij}^k
    \Gamma_{kp}^q-\Gamma_{ip}^k\Gamma_{kj}^q)\vphi_q+(h_{ij}a_p^q-h_{ip}a_j^q)
    \vphi_q &\text{(tangential)} \\
    &+(\Gamma_{ij}^kh_{kp}-\Gamma_{ip}^kh_{kj}+\partial_p h_{ij}-\partial_{j}
    h_{ip})N &\text{(normal)} \\
    =& 0
\end{align*}
\begin{remark}
    If the ambient space is not \(\mathbb{R}^n\), LHS should give curvature of
    the ambient space.
\end{remark}

Note that \(\vphi_i\) and \(N\) are perpendicular, we can split tangential part
and normal part of the equation, and use \(a_{p}^k=-h_{pq}g^{qk}\) we get:

Normal part:
\begin{equation}\label{eq:codazzi}
    \partial_p h_{ij}-\Gamma_{pi}^kh_{kj}=\partial_j h_{ip}-\Gamma_{ji}^kh_{kp}
.\end{equation}

Tangential part: \[
    (h_{ij}h_{pq}-h_{ip}h_{jq})g^{qk}
    =\partial_p \Gamma_{ij}^k-\partial_j\Gamma_{ip}^k
    +\Gamma_{ij}^l\Gamma_{lp}^k-\Gamma_{ip}^l\Gamma_{lj}^k
.\] multiply by \(g^{kr}\) we get
\begin{equation}\label{eq:gauss}
    h_{ij}h_{pq}-h_{ip}h_{jq}=
    g_{qk}\left(\partial_p \Gamma_{ij}^k-\partial_j\Gamma_{ip}^k
    +\Gamma_{ij}^l\Gamma_{lp}^k-\Gamma_{ip}^l\Gamma_{lj}^k\right)
.\end{equation}

\cref{eq:codazzi} is called the Codazzi equation and~\cref{eq:gauss} is called
the Gauss equation. Take \(i=j,p=q\) in Gauss equation, we see LHS becomes \[
    h_{ii}h_{pp}-h_{ip}h_{pi}=\det\begin{bmatrix}
        h_{ii} & h_{ip} \\
        h_{pi} & h_{pp}
    \end{bmatrix}
.\] For 2 dim case, if \(i\neq p\), this gives exactly \(\det\II\). Note RHS
is purely determined by \(I\), hence \(K=\frac{\det\II}{\det I}\) only depends
on \(I\), it is an intrinsic geometric quantity.

{\color{red}\large !} Gaussian curvature is the local geometric invariant of
surfaces.

We look back Codazzi equation, we can add a term as \[
    \partial_{p}h_{ij}-\Gamma_{pi}^kh_{kj}-\boxed{\Gamma_{pj}^kh_{ki}}
    =\partial_{j}h_{ip}-\Gamma_{ji}^kh_{kp}-\boxed{\Gamma_{jp}^kh_{ki}}
.\] In terms of covariant derivative \(\nabla\), this writes \[
    \nabla_p h_{ij}=\nabla_{j} h_{ip}
.\] \ie\ All 3 index of \(\nabla_p h_{ij}\) are symmetric.

\begin{remark}
    We don't need to memorize the Gauss-Codazzi equations precisely. It
    suffices to work it out step by step once we know local parametrization.
    And there is a much more simple form of the equations after we introduced
    notations in Riemannian geometry.
\end{remark}

We also call the Gauss-Codazzi equations are the integrability to solve
the equation of motion~\cref{eq:motion}.

\begin{theorem}[Fundamental theorem of surface theory (local)]
    Let \(U\subset \mathbb{R}^2\) be open, connected set. Given two quadratic
    form \(I=g_{ij}\dd{x^i}\dd{x^j}, \II=h_{ij}\dd{x^i}\dd{x^j}\), \st\ \(I\) is
    positively definite. Moreover, the Gauss-Codazzi equations are satisfied.
    Then there is a surface \(S\) in \(\mathbb{R}^3\) \st\ \(I,\II\) are the 1st
    and 2nd fundamental form of \(S\) with \(U\) a coordinate chart.

    The surface \(S\) is unique up to rigid motion.
\end{theorem}
\begin{proof}
    Skip. (Can be found in Do Carmo's book)
\end{proof}

\newpage
