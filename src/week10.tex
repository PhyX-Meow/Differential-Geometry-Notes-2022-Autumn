\section{Einstein convention}

Recall \(v\in \mathbb{R}^n\), \(v=(v^1,\ldots,v^n)\implies v=\sum_{i=1}^n v^i e_i\).
\begin{enumerate}[(1)]
    \item The summation is taken over \(i=1,2,\ldots,n\);
    \item \(v=\sum_{i=1}^n v^ie_i=\sum_{j=1}^nv^je_j\), expression of \(v\) is
        independent of choice of summation indices \(i\) or \(j\).
    \item One ``\(i\)'' appears as upper index, another ``\(i\)'' appears as lower
        index.
\end{enumerate}

Einstein convention: Whenever there is pair of same letter as upper and a lower
index, then the expression is summation of the index letter from 1 to \(n\),
\(n\) is usually the dimension of vector space / manifold / etc.

\begin{example}
\begin{itemize}
    \item \(W=V\cdot A\), \(A=(A_i^j)_{n\times n}\) matrix, \(v,w\in \mathbb{R}^n\),
        then \[
            W^j=v^i A_i^j=\sum_{i}v^iA_i^j
        .\] 
    \item \[
        g^{ij}\omega_{jk}=\sum_{j=1}^n g^{ij}\omega_{jk}.
    .\] Where \(g^{ij}\) is the \((i,j)\)-entry of inverse matrix \(g^{-1}\) of
    \(g\). \ie\ \[
        g^{ij}g_{jk}=\delta^i_k,
        \quad g^{ij}g_{ij}=\sum_{i=1}^n \sum_{j=1}^{n}g^{ij}g_{ij}
        =\sum_{i=1}^{n}\delta^i_i=n
    .\] 

    We will use Einstein convention from now on.
\end{itemize}
\end{example}

\section{Therema Egregium (Gauss)}

\underline{Goal:} The Gaussian curvature \(K\) depends only on the 1st fundamental
form.

Previously, we have seen once we know local parametrization \(\vphi(x^1,x^2)\) of
a surface \(S\), then \(I,\II\) can be computed and the Gaussian curvature is \[
    K=\frac{\det \II}{\det I}=\frac{eg-f^2}{EG-F^2}
.\] Now, we want to follow the same procedure as we study the Frenet formula of a
curve and try to understand the motion of equation of frame \(\vphi_1,\vphi_2,N\),
where \(\vphi_i=\pdv{\vphi}{x^i}, N=\frac{\vphi_1\times \vphi_2}{|\vphi_1\times 
\vphi_2|}\).

Fix a local parametrization \[
    \vphi\colon U \longrightarrow S\subset \mathbb{R}^3,
    \ (x^1,x^2)\longmapsto \vphi(x^1,x^2)
.\] Then
\begin{align*}
    I&=g_{ij}\dd{x^i}\dd{x^j},\quad g_{ij}=\left<\vphi_i,\vphi_j\right> \\
    \II&=h_{ij}\dd{x^i}\dd{x^j},\quad h_{ij}
.\end{align*}
We shall study the differential equation of \(\{\vphi_i,N\}\) up to 2nd order.
\begin{align*}
    \vphi_{11}&=\Gamma_{11}^1\vphi_1+\Gamma_{11}^1\vphi_2+h_{11}N\\
    \vphi_{12}&=\Gamma_{12}^1\vphi_1+\Gamma_{12}^1\vphi_2+h_{12}N\\
    \vphi_{21}&=\Gamma_{21}^1\vphi_1+\Gamma_{21}^1\vphi_2+h_{21}N\\
    \vphi_{22}&=\Gamma_{22}^1\vphi_1+\Gamma_{22}^1\vphi_2+h_{22}N\\
.\end{align*}
Weingarten equation, \(A=(a_i^j)\)
\[
    \begin{bmatrix}
        N_1 \\ N_2
    \end{bmatrix}=A\begin{bmatrix}
        \vphi_1 \\ \vphi_2
    \end{bmatrix}
.\] Then \[
    a_i^j=-h_{ik}g^{kj}
.\] Lets write above equations as \[
    \begin{cases}
        \vphi_{ij}=\Gamma_{ij}^k\vphi_k+h_{ij}N \\
        N_i=a_i^j=\vphi_j
    \end{cases}
    \implies \text{The only unknown are }\Gamma_{ij}^k
.\] Moreover, \[
    \vphi_{ij}=\vphi_{ji}\implies \Gamma_{ij}^k=\Gamma_{ji}^k
.\] \[
    \left<\vphi_{ij},\vphi_p\right> =\Gamma_{ij}^k \left<\vphi_k,\vphi_p\right> 
    =\Gamma_{ij}^kg_{kl}
.\] So
\begin{gather*}
    \pdv{g_{ip}}{x^j}-\left<\vphi_i,\vphi_{pj}\right> 
    =\left<\vphi_{ij},\vphi_p\right> =\Gamma_{ij}^kg_{kp} \\
    \pdv{g_{jp}}{x^i}-\left<\vphi_j,\vphi_{pi}\right> 
    =\left<\vphi_{ji},\vphi_p\right> =\Gamma_{ji}^kg_{kp} \\
    \pdv{g_{ij}}{x^p}=\left<\vphi_{ip},\vphi_j\right>
    +\left<\vphi_i,\vphi_{jp}\right> 
\end{gather*}
Hence \[
    2\Gamma_{ij}^k g_{kp}=\pdv{g_{jp}}{x^i}+\pdv{g_{ip}}{x^j}-\pdv{g_{ij}}{x_{p}}
.\] \[
    \implies 2\Gamma_{ij}^kg_{kp}g^{pq}
    =(\pdv{g_{jp}}{x^i}+\pdv{g_{ip}}{x^j}-\pdv{g_{ij}}{x_{p}})g^{pq}
.\] We get \[
    \Gamma_{ij}^k=\frac{1}{2}(\pdv{g_{jp}}{x^i}+\pdv{g_{ip}}{x^j}g^{pk}
    -\pdv{g_{ij}}{x_{p}})
.\] 

\begin{remark}
    We multiply \(g^{pk}\) form right because of the choice of row vector. In
    modern convention of the column vector, \[
        \Gamma_{ij}^k=\frac{1}{2}g^{kp}(\pdv{g_{jp}}{x^i}+\pdv{g_{ip}}{x^j}
        -\pdv{g_{ij}}{x_{p}})
    .\] 
\end{remark}
\begin{definition}
    \(\Gamma_{ij}^k\) is called the Christoffel symbols. They're uniquely determined
    by the 1st fundamental form.
\end{definition}


\newpage
