\section{Method of Moving Frame}
Let \(\vphi(x^1,x^2)\) be a local coordinate of \(S\hookrightarrow\mathbb{R}^3\).
We have already studied the \emph{equation of motion} of coordinate frame \(\{\vphi_1,
\vphi_2,\vec{N}\}\). Recall that we obtained: 
\begin{equation}\label{eq:recall-eq-motion}
    \begin{cases}
        \vphi_{ij}=\Gamma_{ij}^k\vphi_k+h_{ij}N \\
        N_p=a_p^q\vphi_q
    \end{cases}
.\end{equation}
Where \[
    \begin{cases}
        \Gamma_{ij}^k=\frac{1}{2}g^{kl}(\partial_i g_{jl}+\partial_j g_{il}-\partial_l
        g_{ij}) \\
        a_p^q=-h_{pk}g^{kq}
    \end{cases}
.\] And \(g,h\) be the first and second fundamental forms.

Furthermore, the Gauss-Codazzi equation is the integrability condition to solve
\cref{eq:recall-eq-motion}. This phenomenon can be view more generally as \[
    C^0\xlongrightarrow{\dd}C^1\xlongrightarrow{\dd}C^2
.\] \(\dd^2=0\iff\) integrability condition. 

\subsection{Darboux moving frame (Local orthonormal frame)}
\subsubsection{Curve case}
Let \(\alpha(s)\) be a space curve, \(s\) be the arc-length 
parameter. Recall we have learned the Frenet frame \(\{T,N,B\}\), which is orthonormal
along \(\alpha(s)\) where \[
    T(s)=\alpha'(s),\quad N(s)=\frac{\alpha''(s)}{|\alpha''(s)|},\quad
    B(s)=T(s)\wedge N(s)
.\] And the equation of motion is \[
    \dv{s}\begin{pmatrix}
        T(s) \\ N(s) \\ B(s)
    \end{pmatrix}=\begin{pmatrix}
        0 & k(s) & 0 \\
        -k(s) & 0 & -\tau(s) \\
        0 & \tau(s) & 0
    \end{pmatrix}\begin{pmatrix}
        T(s) \\ N(s) \\ B(s)
    \end{pmatrix}
.\] Recall that we derived the equation from geometry.

Now let \(e_1,e_2,e_3\) be any orthonormal frame along \(\alpha(s)\). Let's fix
\(e_1(s)=\alpha'(s)\) and take differential \[
    \dd{e_i}(s)=e_i'(s)\dd{s}\quad \text{(vector valued 1-form)}
.\] Note \(e_i'(s)\) is still a vector field, let \[
    e_i'(s)=\sum_{j=1}^{3}b_i^j(s)e_i(s)
.\] Then \[
    \dd{e_i}(s)=b_i^j(s)\underbrace{\dd{s}}_{\text{1-form}}
    \underbrace{e_j(s)}_{\text{vector}}
.\] About coefficients \(b_i^j\), since \(\left<e_i,e_j\right> =0\), \(\implies b_i^j
+b_j^i=0\), \ie\ \[
    \begin{cases}
        e_1'=b_1^2e_2+b_1^3e_3 \\
        e_2'=-b_1^2e_1+b_2^3e_3 \\
        e_3'=-b_1^3e_1-b_2^3e_2
    \end{cases}
.\] By fixing \(e_1=\alpha'\), any other orthonormal frame \(\{e_1,\tilde{e}_2,
\tilde{e}_3\}\) is obtained by rotating \(e_2\) and \(e_3\). \ie\ \[
    \begin{pmatrix}
        e_2 \\ e_3
    \end{pmatrix}=\begin{pmatrix}
        \cos\theta & \sin\theta \\
        -\sin\theta & \cos\theta
    \end{pmatrix}\begin{pmatrix}
        e_2 \\ e_3
    \end{pmatrix}
.\] Then we have \[
    (b_1^2,b_1^3)=(\tilde{b}_1^2,\tilde{b}_1^3)\begin{pmatrix}
        \cos\theta & \sin\theta \\
        -\sin\theta & \cos\theta
    \end{pmatrix}
.\] By choosing \(\theta\), we can let \(\tilde{b}_1^3=0\), then \[
    \begin{cases}
        \tilde{e}_1'=\tilde{b}_1^2 \\
        \tilde{e}_2'=-\tilde{b}_1^2\tilde{e}_1+\tilde{b}_2^3\tilde{e}_3 \\
        \tilde{e}_3'=-\tilde{b}_2^3\tilde{e}_2
    \end{cases}
.\] Since \(\alpha''(s)=\tilde{e}_1'(s)=\tilde{b}_1^2\tilde{e}_2=k(s)N(s)\),
we see \(\tilde{b}_1^2=k\) up to a sign. If we further let \(\tilde{e}_2=N\),
then \(\tilde{e}_3=\tilde{e}_1\wedge \tilde{e}_2\) is just \(B\). Hence
\(\tilde{b}_2^3=\tau\).

\subsubsection{Surface case}
Our next goal is to study the 1st and 2nd fundamental form
in terms of local orthonormal frame.

\textbf{Existence:} \\
Let \(\vphi(x^1,x^2)\) be a local coordinate chart on \(U\subset S\), then \(TS\big|_U
=\Span{\vphi_1,\vphi_2}\). Let \[
    e_1=\frac{\vphi_1}{|\vphi_1|},\quad e_2=\frac{\vphi_2-\left<\vphi_2,e_1\right> e_1}
    {|\vphi_2-\left<\vphi_2,e_1\right> e_1|}
.\] Then \(\{e_1,e_2\}\) is an orthonormal frame. Moreover, let \(e_3=e_1\wedge e_2\),
then \(e_3\) is unit normal vector field on \(U\). \(\{e_1,e_2,e_3\}\) gives
an orthonormal frame of \(\mathbb{R}^3\) on \(U\).

For any other orthonormal frame \(\{\tilde{e}_1,\tilde{e}_2\}\) on \(U\), it differs
from \(\{e_1,e_2\}\) by an \(SO(2)\) matrix \[
    R(x^1,x^2)=\begin{pmatrix}
        \Gamma_{11}(x^1,x^2) & \Gamma_{12}(x^1,x^2) \\
        \Gamma_{21}(x^1,x^2) & \Gamma_{22}(x^1,x^2)
    \end{pmatrix}
.\] And for any \(p\in U\), \(\det R(p)=1\).

Now, let's assume \(\{e_1,e_2\}\) be any orthonormal frame on \(U\), \(e_3=e_1\wedge 
e_2\). Then there exists a linear transformation \(T=T(x^1,x^2)\in GL(2)\), such that
\[
    \begin{pmatrix}
        \vphi_1 \\ \vphi_2
    \end{pmatrix}=T\begin{pmatrix}
        e_1 \\ e_2
    \end{pmatrix}\implies \begin{cases}
        \vphi_1=t_1^1e_1+t_1^2e_2 \\
        \vphi_2=t_2^1e_1+t_2^2e_2
    \end{cases}
.\] Note \(\vphi=\vphi(x^1,x^2)\), 
\begin{flalign*}
    \implies\dd{\vphi}&=\vphi_1\dd{x^1}+\vphi_2\dd{x^2} &
    \implies &\dd{\phi} \text{ is a vector valued differential 1-form on }U \\
    &&&\text{\ie\ }\dd{\vphi}\in \Gamma(U,\vphi^*(T\mathbb{R}^3\otimes T^*U))
    \text{(basis change)}&=(t_1^1e_1+t_1^2e_2)\dd{x^1}+(t_2^1e_1+t_2^2e_2)\dd{x^2} \\
    &=(t_1^1\dd{x^1}+t_2^1\dd{x^2})e_1+(t_1^2\dd{x^1}+t_2^2\dd{x^2})e_2 \\
    &=\omega^1e_1+\omega^2e_2
.\end{flalign*}
Where \((\omega^1,\omega^2)=(\dd{x^1},\dd{x^2})\cdot T\). \ie\ in terms of orthonormal
frame \(\{e_1,e_2\}\) on \(U\), \[
    \dd{\vphi}=\omega^1e_1+\omega^2e_2
.\] 

Now, we study the 1st and 2nd fundamental form in terms of \(\{e_1,e_2\}\).
Recall that \[
    I=(\dd{s}^2_{\mathbb{R}^3})\big|_S=\delta_{ij}\dd{y^i}\dd{y^j}
    =(\dd{y^1})^2+(\dd{y^2})^2+(\dd{y^3})^2
.\] Note \(\vphi=\vphi(x^1,x^2)=(y^1(x^1,x^2),y^2(x^1,x^2),y^3(x^1,x^2))\). Then \[
    \dd{\vphi}=(\dd{y^1},\dd{y^2},\dd{y^3})
.\] Hence \[
    I=\left<\dd{\vphi},\dd{\vphi}\right> _{\mathbb{R}^3}
.\] Plug in \(\dd{\vphi}=\omega^1e_1+\omega^2e_2\), we have \[
    I=\left<\omega^1e_1+\omega^2e_2,\omega^1e_1+\omega^2e_2\right>_{\mathbb{R}^3}
    =(\omega^1)^2+(\omega^2)^2
.\]
\begin{remark}\hfill
\begin{itemize}
\item \(\{\omega^1,\omega^2\}\) are dual coframe of \(\{e_1,e_2\}\), \ie\ \(\omega^i
    (e_j)=\delta^i_j\).
\item \(\{\omega^1,\omega^2\}\) is orthonormal frame on \(T^*U\).
\end{itemize}
\end{remark}

Next, we study the 2nd fundamental form in terms of \(\{e_1,e_2\}\) and \(e_3=e_1
\wedge e_2\). \(\dd{e_i}\) is vector valued 1-forms on \(U\), and \(\{e_1,e_2,e_3\}\)
is basis for \(\mathbb{R}^3\), so \[
    \dd{e_i}=\omega_i^je_j
.\] Where \(w_i^j\) are differential 1-forms on \(U\). Moreover \[
    \omega_i^j=\left<\dd{e_i},e_j\right> =-\left<e_i,\dd{e_j}\right> =-\omega_j^i
.\] Since \(e_3\) is the unit normal, \(-\dd{e_3}\) is just the Weingarten map, hence
\begin{align*}
    \II&=-\left<\dd{e_3},\dd{\vphi}\right> =-\left<\omega_3^1e_1+\omega_3^2e_2,
    \omega^1e_1+\omega^2e_2\right> \\
    =-\omega_3^1\omega^1-\omega_3^2\omega^3=\omega_1^3\omega^1+\omega_2^3\omega^2
.\end{align*}

In summary, if \(\{e_1,e_2,e_3\}\) is orthonormal frame of \(S\) on \(U\), with
\(e_1,e_2\in T_U S\), \(e_3\in N_U S\), and \(\omega^1,\omega^2\in T_U^*S\) is
dual coframe of \(\{e_1,e_2\}\), then \[
    \begin{cases}
        \dd{e_i}=\omega_i^je_j \\
        \omega_i^j+\omega_j^i=0
    \end{cases},\quad \begin{cases}
        I=(\omega^1)^2+(\omega^2)^2 \\
        \II=\omega_1^3\omega^1+\omega_2^3\omega^2
    \end{cases}
.\] 
\begin{remark}\hfill
\begin{itemize}
\item Here \(\omega^i,\omega_j^k\) are covariant 1-tensors, \[
    \omega^i\omega_i^k=\frac{1}{2}(\omega^i\otimes\omega_i^k+\omega_i^k\otimes\omega^i)
    \quad \text{(symmetrization of }\omega^i,\omega_j^k\text{)}
.\] 
\item Since \(\omega_1^3,\omega_2^3\) are differential 1-forms on \(U\), \ie\ \(
    \omega_1^3,\omega^3\in T_U^*S=\Span\{\omega^1,\omega^2\}\), we can let \[
        \begin{cases}
            \omega_1^3=A_{11}\omega^1+A_{12}\omega^2 \\
            \omega_2^3=A_{21}\omega^1+A_{22}\omega^2
        \end{cases}
    .\] Then \[
        \II=(\omega^2,\omega^2)\begin{pmatrix}
            A_{11} & A_{12} \\
            A_{21} & A_{22}
        \end{pmatrix}\begin{pmatrix}
            \omega^1 \\ \omega^2
        \end{pmatrix}
    .\] In particular, since \(\II\) is symmetric, \(A_{12}=A_{21}\).
\end{itemize}
\end{remark}

\subsection{Exterior Derivative \& Cartan's Structure Equation}
\subsubsection{Differential forms}
Reference: Kodaira's book \emph{Complex manifolds and Deformation of Complex Structures}

Let \(M\) be a smooth manifold of dimension \(n\), let \(\{U_\alpha\}\) be a coordinate covering of \(M\). Define
\begin{itemize}
\item \(\Omega^0(M)=C^\infty(M)=\{\text{smooth functions on }M\}\)
    \(f\in \Omega^0(M)\) is called a \emph{smooth 0-form}.
\item \(\Omega^1(M)=\Gamma(M,T^*M)=\{\text{smooth co-vector fields on }M\}\).
    \(\omega\in \Omega^1(M)\) is called a \emph{smooth 1-form}
\end{itemize}

Note on each \(U_\alpha\), \[
    \omega\big|_{U_\alpha}=a_i\dd{x_\alpha^i}
\] where \(a_i=a_i(x_\alpha^1,\ldots,x_\alpha^n)\) are smooth functions on
\(U_\alpha\). On \(U_\alpha\cap U_\beta\), \[
    \omega\big|_{U_\alpha\cap U_\beta}=a_i\dd{x_\alpha^i}=b_j\dd{x_\beta^j}
.\] Then \[
    \begin{pmatrix}
        a_1 \\ \vdots \\ a_n
    \end{pmatrix}=\begin{pmatrix}
        \displaystyle\pdv{(x_\beta^1,\ldots,x_\beta^n)}{(x_\alpha^1,\ldots,x_\alpha^n)}
    \end{pmatrix}\begin{pmatrix}
        b_1 \\ \vdots \\ b_n
    \end{pmatrix}
.\] \ie\ a smooth 1-form on \(S\) is obtained by gluing local differential 1-forms
using the transition functions.

Using the similar idea, we define a smooth 2-form \(\eta\in \Omega^2(M)\) as:
\begin{itemize}
\item On each \(U_\alpha\), \(\eta_\alpha=\frac{1}{2!}f_{\alpha ij}\dd{x_\alpha^i}
    \wedge \dd{x_\alpha^j}\). Where \(f_{\alpha ij}\) are smooth functions on
    \(U_\alpha\), \(\dd{x_\alpha^i}\wedge \dd{x_\alpha^j}=-\dd{x_\alpha^j}\wedge 
    \dd{x_\alpha^i}\).
\item On \(U_\alpha\cap U_\beta\), \(\eta_\alpha=\eta_\beta\), then \[
        \eta=\frac{1}{2!}f_{\alpha ij}\dd{x_\alpha^i}\wedge \dd{x_\alpha^j}
        =\frac{1}{2!}f_{\beta ij}\dd{x_\beta^i}\wedge \dd{x_\beta^j}
    \] is called a \emph{differential 2-form} on \(M\). Note that \[
        \eta_\alpha=\eta_\beta\implies 
        f_\alpha=f_\beta \det(\pdv{(x_\beta^1,\ldots,x_\beta^n)}
        {(x_\alpha^1,\ldots,x_\alpha^n)})
    .\] 
\item Let \(\Omega^2(M)=\Gamma(M,\wedge^2T^*M)=\{\text{smooth 2-forms on }M\}\).
\end{itemize}

In general, a smooth \(k\)-form \(\eta\in \Omega^k(M)\) is defined as:
\begin{itemize}
\item On each \(U_\alpha\), \(\eta_\alpha=\frac{1}{2!}f_{\alpha i_1\cdots i_k}
    \dd{x_\alpha^i}\wedge\cdots\wedge\dd{x_\alpha^j}\).
    Where \(f_{\alpha i_1\cdots i_k}\) are smooth functions on \(U_\alpha\),
    \(\dd{x_\alpha^i}\wedge \dd{x_\alpha^j}=-\dd{x_\alpha^j}\wedge\dd{x_\alpha^i}\).
\item On \(U_\alpha\cap U_\beta\), \(\eta_\alpha=\eta_\beta\), then \[
        \eta=\frac{1}{2!}f_{\alpha ij}\dd{x_\alpha^i}\wedge \dd{x_\alpha^j}
        =\frac{1}{2!}f_{\beta ij}\dd{x_\beta^i}\wedge \dd{x_\beta^j}
    \] is called a \emph{differential 2-form} on \(M\).
\end{itemize}

\subsubsection{Wedge Product}
Let \(\omega\in \Omega^p(M),\eta\in \Omega^q(M)\), locally \[
    \omega=\frac{1}{p!}\omega_{i_1\cdots i_p}\dd{x^{i_1}}\wedge \cdots \dd{x^{i_p}},
    \quad\eta=\frac{1}{q!}\eta_{j_1\cdots j_q}\dd{x^{j_1}}\wedge \cdots \dd{x^{j_q}}
.\] Then define \[
    \omega\wedge\eta=\frac{1}{p!q!}\omega_{i_1\cdots i_p}
    \eta_{j_1\cdots j_q}\dd{x^{i_1}}\wedge \cdots \dd{x^{i_p}}\wedge
    \dd{x^{j_1}}\wedge \cdots \dd{x^{j_q}}
.\] 

\noindent\textbf{Example.}
\(p=q=1\), \(\omega=\omega_i\dd{x^i},\eta=\eta_j\dd{x^j}\) \[
    \omega\wedge \omega_i\eta_j\dd{x^i}\wedge\dd{x^j}
    =\frac{1}{2}\sum_{i<j}(\omega_i\eta_j-\omega_j\eta_i)\dd{x^i}\wedge \dd{x^j}
.\] If \(\dim M=2\), \(\omega=\omega_1\dd{x^1}+\omega_2\dd{x^2}\), \(\eta=\eta_1
\dd{x^1}+\eta_2\dd{x^2}\), \[
    \omega\wedge \eta=\frac{1}{2}(\omega_1\eta_2-\omega_2\eta_1)\dd{x^1}\wedge
    \dd{x^2}
.\] And \(\omega\wedge \eta=-\eta\wedge \omega\).

\subsubsection{Exterior derivative}
\begin{itemize}
\item \(\dd\colon \Omega^0(M)\longrightarrow \Omega^1(M),f\longmapsto\dd{f}\).
    Locally, \(\dd{f}=\pdv{f}{x^i}\dd{x^i}\).
\item \(\dd\colon \Omega^1(M)\to \Omega^2(M),\omega\mapsto \dd{\omega}\).
    Locally, write \(\omega=\omega_i\dd{x^i}\), then
    \begin{align*}
        \dd{\omega}&=\dd{\omega_i}\wedge \dd{x^i}=\pdv{\omega_i}{x^j}\dd{x^i}
        \wedge \dd{x^i} \\
        &=\frac{1}{2}(\pdv{\omega_i}{x^j}-\pdv{\omega_j}{x^i})\dd{x^j}\wedge\dd{x^i}\\
        &=\sum_{j<i}(\pdv{\omega_i}{x^j}-\pdv{\omega_j}{x^i})\dd{x^j}\wedge\dd{x^i}
    .\end{align*}
\item Generally, for a \(k\)-form \(\eta=\frac{1}{k!}\eta_{i_1\cdots i_k}\dd{x^{i_1}}
    \wedge \cdots \wedge \dd{x^{i_k}}\), \[
        \dd{\eta}=\frac{1}{(k+1)!}\sum_{i_0,\ldots,i_k}(\dd{\eta})_{i_0\cdots i_k}
        \dd{x^{i_0}}\wedge \dd{x^{i_1}}\wedge \cdots \wedge \dd{x^{i_k}}
    .\] Then we have \[
        (\dd{\eta})_{i_0i_1\cdots i_k}=\sum_{s=0}^k(-1)^s
        \pdv{x^{i_s}}\eta_{i_0i_1\cdots i_{s-1}i_{s+1}\cdots i_k}
    .\] 
\end{itemize}

\noindent\textbf{Exercise:}
\begin{enumerate}[(1)]
\item Check that \(\dd\colon \Omega^n(M)\to 0\), \ie\ for any \(n\)-form \(\eta\),
    \(\dd{\eta}=0\).
\item Check that \(\dd^2=0\). Then we have \[
    \cdots\longrightarrow\Omega^{k-1}(M)\xrr{\dd}\Omega^k(M)
    \xrr{\dd}\Omega^{k+1}(M)\longrightarrow\cdots\qquad (*)
    \] This implies \(\img(\dd\colon \Omega^k\to \Omega^{k+1})\subset \ker(\dd\colon
    \Omega^k\to \Omega^{k+1})\), hence \((*)\) is a chain complex. Then from Algebraic
    Topology, we can define the \(k\)-th cohomology group \[
        H_{\text{dR}}^k(M)=\ker \dd/\img \dd
    ,\] which is called the \(k\)-th de Rham cohomology group of \(M\). Moreover
    \(b_k=\dim H_{\text{dR}}^k(M)\) is called the \(k\)-th Betti number of \(M\).
\end{enumerate}

\begin{theorem}[de Rham]
    Let \(M\) be a smooth manifold, then for all \(k\), \[
        H_{\text{dR}}^k(M)\cong H^k(M,\mathbb{R})
    .\] Where the RHS is the \(k\)-th singular cohomology groups of \(M\).
\end{theorem}

\begin{enumerate}[(3)]
\item The exterior derivative and the wedge product satisfy \[
    \dd{(\omega\wedge \eta)}=\dd{\omega}\wedge \eta+(-1)^p\omega\wedge \dd{\eta}
.\] For \(\omega\in \Omega^p(M)\), \(\eta\in \Omega^q(M)\).
\end{enumerate}

\subsection{Cartan's Structure Equation}
Let \(\vphi(x^1,x^2)\) be local coordinate on \(U\), \(T_US=\Span\{e_1,e_2\}\),
orthonormal basis, \(e_3=e_1\wedge e_2\). Then \[
    \begin{cases}
        \dd{\vphi}=\omega^1e_1+\omega^2e_2 & (1)\\
        \dd{e_i}=\omega_i^1e_1+\omega_i^2e_2+\omega_i^3e_3 & (2)
    \end{cases},\qquad\omega_i^j=-\omega_j^i
.\]

Take exterior derivative of (1),
\begin{align*}
    0&=\dd{\dd{\vphi}}=\dd{\omega^1}e_1-\omega^1\wedge\dd{e_1}+\dd{\omega^2}e_2
    -\omega^2\wedge\dd{e_2} \\
    &=\dd{\omega^1}e_1-\omega^1\wedge(\omega_1^2e_2+\omega_1^3e_3)+\dd{\omega^2}e_2
    -\omega^2\wedge(\omega_2^1e_1+\omega_2^3e_3) \\
    &=(\dd{\omega^1}-\omega^2\wedge \omega_2^1)e_1+(\dd{\omega^2}-\omega^1\wedge 
    \omega_1^2)e_2-(\omega^1\wedge \omega_1^3+\omega^2\wedge \omega_2^3)e_3
.\end{align*}
Hence
\begin{equation}\label{eq:structure1}
    \begin{cases}
        \dd{\omega^1}=\omega^2\wedge \omega_2^1 \\
        \dd{\omega^2}=\omega^1\wedge \omega_1^2 \\
        \omega^1\wedge \omega_1^3+\omega^2\wedge \omega_2^3=0
    \end{cases},\qquad \omega_2^1+\omega_1^2=0
.\end{equation}
Note \[
    \begin{cases}
        \omega_1^3=A_{11}\omega^1+A_{12}\omega^2 \\ 
        \omega_2^3=A_{21}\omega^1+A_{22}\omega^2
    \end{cases}\implies \boxed{\begin{aligned}
        &\,\omega^1\wedge \omega_1^3+\omega^2\wedge \omega_2^3 \\
        =&\,A_{12}\omega^1\wedge \omega^2+A_{21}\omega^2\wedge \omega^1 \\
        =&\,(A_{12}-A_{21})\omega^1\wedge \omega^2
    \end{aligned}}
\] Hence \(A_{12}=A_{21}\)

Take exterior derivative of (2),
\begin{align*}
    0&=\dd{\dd{e_i}}=\dd{(\omega_i^je_j)}=\dd{\omega_i^j}e_j-\omega_i^j\wedge\dd{e_j}\\
    &=\dd{\omega_i^j}e_j-\omega_i^j\wedge (\omega_j^ke_k) \\
    &=(\dd{\omega_i^j}-\omega_i^k\wedge \omega_k^j)e_j
.\end{align*}
Hence \(\dd{\omega_i^j}=\omega_i^k\wedge \omega_k^j\). Note \(\omega_i^j+\omega_j^i=0
\implies\) only \(\omega_1^3,\omega_1^3,\omega_2^3\) is non-zero. Then\
\begin{equation}\label{eq:structure2}
    \begin{cases}
        \dd{\omega_1^2}=\omega_1^3\wedge \omega_3^2 & \text{(Gauss)} \\
        \dd{\omega_1^3}=\omega_1^3\wedge \omega_2^3 & \text{(Codazzi I)} \\
        \dd{\omega_2^3}=\omega_2^1\wedge \omega_1^3 & \text{(Codazzi II)}
    \end{cases}
.\end{equation}

\Cref{eq:structure1} and \cref{eq:structure2} are called Cartan's 1st and 2nd
structure equations.

Let's further compute the first equation in \cref{eq:structure2}.
\begin{align*}
    \dd{\omega_1^2}&=\omega_1^3\wedge \omega_3^2 \\
    &=(A_{11}\omega^1+A_{12}\omega^2)\wedge (-(A_{21}\omega^1+A_{22}\omega^2)) \\
    &=-(A_{11}A_{22}-A_{12}A_{21})\omega^1\wedge\omega^2 \\
    &=-K\omega^1\wedge \omega^2
.\end{align*}
Hence if we only consider the surface \(S\) itself without the ambient \(\mathbb{R}^3\)
then Cartan's structure equation is
\begin{equation}\label{eq:cartan}
    \begin{cases}
        \dd{\omega^1}=\omega^2\wedge \omega_2^1 \\
        \dd{\omega^2}=\omega^1\wedge \omega_1^2 \\
        \dd{\omega_1^2}=-K\omega^1\wedge \omega^2
    \end{cases}
.\end{equation}

\noindent\textbf{Exercise.} Let \(\dd{s}^2=\dd{r}^2+\vphi(r)^2\dd{\theta}^2\),
check that \[
    K=-\frac{\vphi''(r)}{\vphi(r)}
.\] 

In Riemannian Geometry, we can also derive Cartan's structure equation.

Let \((M,g)\) be Riemannian manifold, \(\nabla\) be the Levi-Civita connection.
Choose \(\{e_1,\ldots,e_n\}\) be local orthonormal frame and \(\{\omega^1,\ldots,
\omega^n\}\) be the dual frame.

For any vector field \(X\), let \(\nabla_X e_j=\omega_j^i(X)e_i\), where \(\omega_j^i\)
is the connection 1-form, satisfying \(\omega_i^j+\omega_j^i=0\).
For any vector field \(X,Y\), \[
    R(X,Y)e_j=\Omega_j^i(X,Y)e_i
.\] Where \(\Omega_j^i\) is the curvature 2-form, satisfying \[
    \begin{cases}
        \Omega_j^i(X,Y)+\Omega_j^i(Y,X)=0 \\ 
        \omega_j^i=\frac{1}{2}\tensor{R}{_k_l^i_j}\omega^k\wedge \omega^l
    \end{cases}
.\] Then the Cartan's structure equation is
\begin{equation}\label{eq:cartan-ndim}
    \dd{\omega^i}=\omega^j\wedge \omega_j^i \\
    \dd{\omega_j^i}=\omega_j^k\wedge \omega_k^i+\omega_j^i
.\end{equation}

Next, we give an more precise explanation of the equation of motion. In \[
    \begin{cases}
        \dd{e_1}=\omega_1^2e_2+\omega_1^3e_3 \\
        \dd{e_2}=\omega_2^1e_1+\omega_2^3e_3 \\
        \dd{e_3}=\omega_3^1e_1+\omega_3^2e_2
    \end{cases}
.\] \(\{e_1,e_2\}\) are tangent vector fields of \(S\). In first and second equations,
we only consider the tangential components and denote \[
    \begin{cases}
        \nabla e_1=\omega_1^2e_2 \\
        \nabla e_2=\omega_2^1e_1
    \end{cases}
.\] \ie\ for any vector field \(X\), \[
    \begin{cases}
        \nabla_X e_1=\omega_1^2(X)e_2 \\
        \nabla_X e_2=\omega_2^1(X)e_1
    \end{cases}
.\] Note as functions, \(\omega_1^2(X)=-\omega_2^1(X)\).

\(e_3\) is the unit normal (Gauss map) \(\implies-\dd{e_3}\) is Weingarten map.
Recall that we let \[
    \omega_1^3=A_{11}\omega^1+A_{12}\omega^2 \\
    \omega_2^3=A_{21}\omega^1+A_{22}\omega^2
.\] Then \[
    \II(e_i,e_j)=\left<-\dd{e_3}(e_i),e_j\right> =\omega_i^3(e_j)=A_{ij}
\] for \(i,j\in \{1,2\}\). This explains \(\omega_1^3\) and \(\omega_2^3\) are
just component of Weingarten map in terms of \(\{e_1,e_2\}\).

