\begin{definition}[Gaussian curvature \& mean curvature]
    Follow the definition of principal curvature, we define
    \begin{itemize}
        \item Gaussian curvature: \(K=k_1k_2\).
        \item Mean curvature: \(H=\frac{k_1+k_2}{2}\).
    \end{itemize}
\end{definition}

\begin{remark}
\begin{enumerate}[(1)]
    \item These two curvatures are very import ant in understanding the surface.
    \item Above definition is in terms of orthonormal basis \(\{e_1,e_2\}\) on
        \(T_p S\), at each \(p\in S\). \[
            \dd{N}_p \begin{bmatrix}
                e_1 \\ e_2
            \end{bmatrix}=\begin{bmatrix}
                -k_1 & 0 \\
                0 & -k_2
            \end{bmatrix}\begin{bmatrix}
                e_1 \\ e_2
            \end{bmatrix}
            \qquad
            K \sim \det,\ H \sim\mathrm{trace}
        .\] 
\end{enumerate}
\end{remark}

\begin{exercise}
    Find the expression of \(K\) and \(H\) in arbitrary parametrization.
    The answer is \[
        K=\frac{\det\II}{\det I}=\frac{eg-f^2}{EG-F^2},\quad
        H=\frac{1}{2}\tr_I\II=\frac{1}{2}\frac{eG-2fF+gE}{EG-F^2}
    .\] 
\end{exercise}

Once we know the Gaussian curvature \(K\) and mean curvature \(H\), by
the fact that \(K=\det A,H=-\frac{1}{2}\tr A\), we can write the 
characteristic polynomial of \(A\): \[
    \det(\lambda I-A)=\lambda^2-\tr A \lambda+\det A
    =\lambda^2+2H\lambda+K
.\] Since the principal curvature \(k=-\lambda\), we have \[
    k^2-2Hk+K=0
.\] And principal curvatures are: \[
    k=H\pm \sqrt{H^2-K}
.\] 

\begin{remark}\hfill
\begin{enumerate}[(1)]
    \item 
    The Gaussian curvature is an ``intrinsic'' curvature, it's only determined
    by the surface itself. The Gaussian elegant theorem tell us that the Gaussian
    curvature is only determined by 1st fundamental form. This already sheds light
    on the Riemannian Geometry. (We'll see the theorem later). The most beautiful
    result in surface theory is the Gauss-Bonnet's theorem: If \(S\) is an oriented
    compact surface without boundary, then \[
        \int_{S}K = 2\pi\chi(S)=2\pi(2-2g)
    .\]
    \item 
    The mean curvature is an ``extrinsic curvature''. It depends on the ambient
    space. (Here, our ambient space is \(\mathbb{R}^3\)). One of important problems
    in Differential Geometry is studying the surfaces with vanishing mean curvature.
    Such surface is called ``minimal surface''. (We will explain ``minimal'' later).
    This problem heavily depends on PDE theory (of 2nd order elliptic type).
\end{enumerate}
\end{remark}

Here, let's have a further understanding of principal curvature \& principal
direction from analysis point of view. We have seen:
\begin{enumerate}[(1)]
    \item Normal curvature at \(p\in S\): 
        \begin{align*}
            k_n\colon \mathbb{S}^1(T_p S) &\longrightarrow \mathbb{R} \\
            v &\longmapsto k_n(v)
        .\end{align*}
    \item Principal direction is at which \(k_n\) attains maximum / minimum.
    \item \(\forall\,v\) not necessary a unit vector, then \[
            k_n(v)=\frac{\II_p(v,v)}{I_p(v,v)}
        .\] 
        Let \(v=v_1\vphi_1+v^2\vphi_2\), where \(\vphi(u^1,u^2)\) is a local
        parametrization. Then \[
            k_n(v)=\frac{(v^1)e^2+2v^1v^2f+(v^2)^2g}{(v^1)^2E+2v^1v^2F+(v^2)^2G}
        .\] WLOG assume \(v^1\neq 0\), \(\lambda=\frac{v^2}{v^1}\), then \[
            k_n(\lambda)=\frac{e+2f\lambda+g\lambda^2}{E+2F\lambda+G\lambda^2}
        .\] 
    \item Since principal curvatures are critical values of \(k_n\), \(k_n'(\lambda)
        =0\), 
        \begin{equation}\label{analysis-kn-1}
            \iff (2f+2g\lambda)(E+2F\lambda+G\lambda^2)-(2F+2G\lambda)
            (e+2f\lambda+g\lambda^2)=0
        .\end{equation}
        \ie\ \[
            \frac{e+2f\lambda+g\lambda^2}{E+2F\lambda+G\lambda^2}
            =\frac{f+g\lambda}{F+G\lambda}
        .\] Hence \[
            k_n=\frac{f+g\lambda}{F+G\lambda}
        .\] On the other hand, \cref{analysis-kn-1} also implies
        \begin{equation}\label{analysis-kn-2}
        \begin{split}
            &(f+g\lambda)(E+F\lambda)+\lambda(f+g\lambda)(F+G\lambda) \\
            =&(F+G\lambda)(e+f\lambda)+\lambda(F+G\lambda)(f+g\lambda)
        .\end{split}\end{equation}
        \ie\ \[
            \frac{f+g\lambda}{F+G\lambda}=\frac{e+f\lambda}{E+F\lambda}
        .\] Hence \[
            k_n=\frac{f+g\lambda}{F+G\lambda}=\frac{e+f\lambda}{E+F\lambda}
            \iff \text{principal curvature}
        .\] Then \[
            \begin{cases}
                f+g\lambda=k_n(F+G\lambda) \\
                e+f\lambda=k_n(E+F\lambda)
            \end{cases}\implies \begin{cases}
            f-Fk_n=(Gk_n-g)\lambda \\
            e-Ek_n=(Fk_n-f)\lambda
            \end{cases}
        .\] This linear system has solution \(\lambda\iff \) \[
            \det\begin{bmatrix}
                e-k_n E & f-k_n F \\
                f-k_n F & g-k_n G
            \end{bmatrix}=0
        .\] This gives an equation to solve \(k_n\).
        To see the principal direction, \ie\ solving \(\lambda\),
        \begin{align*}
            \text{\cref{analysis-kn-2}}
            &\implies (gF-fG)\lambda^2+(gE-eG)\lambda+(fE-eF)=0 \\
            &\iff \det\begin{bmatrix}
                \lambda^2 & -\lambda & 1 \\
                E & F & G \\
                e & f & g
            \end{bmatrix}=0
        .\end{align*}
\end{enumerate}

Next, we would like to introduce several special points on a surface by using
curvatures introduced previously.

\begin{definition}[Classification of points on surface]\hfill\\
    A point \(p\) on a regular surface \(S\) is called a
\begin{enumerate}[(1)]
    \item Elliptic point, if \(K_p>0\).
        (All normal section have the same normal vectors)
    \item Hyperbolic point, if \(K_p<0\).
        (There exist two normal sections with opposite normal vectors)
    \item Parabolic point if \(K_p=0\) but \(\dd{N}_p\neq 0\).
        (One of principal directions is ``flat'')
    \item Planar point, if \(\dd{N}_p=0\), \ie\ \(k_1=k_2=0\).
    \item Umbilical point, if \(k_1=k_2\).
\end{enumerate}
\end{definition}

\begin{remark}\hfill
\begin{enumerate}[(1)]
    \item Apparently, umbilical points can only occur at elliptic point or
        planar point.
    \item At umbilical points, \(\II=kI\).
    \item On a minimal surface, \(H=0\implies k_1=-k_2\), so \(K\le 0\), and there
        is no elliptic point.
\end{enumerate}
\end{remark}

\begin{definition}
    \(S\) is called totally umbilical, if all points are umbilical. For example,
    both plane and sphere are totally umbilical.
\end{definition}

\begin{theorem}
    If \(S\hookrightarrow\mathbb{R}^3\) is totally umbilical and connected. Then
    \(S\) is either contained in a plane or a sphere.
\end{theorem}
\begin{proof}
    Umbilical \(\implies \) at each point, \(k_1=k_2=k\), then \(\II_p=kI_p\).

    Let  \(\vphi(u,v)\) be a local parametrization near \(p\). By Weingarten
    equation, \[
        \begin{bmatrix}
            \vb{n}_u \\ \vb{n}_v
        \end{bmatrix}=A\begin{bmatrix}
            \vphi_U \\ \vphi_v
        \end{bmatrix}=\begin{bmatrix}
            -k & 0 \\
            0 & -k
        \end{bmatrix}\begin{bmatrix}
            \vphi_u \\ \vphi_v
        \end{bmatrix},\quad A=\II\cdot I^{-1}
    .\] Hence \[
        \begin{cases}
            \vb{n}_u=-k\vphi_u \\
            \vb{n}_v=-k\vphi_v
        \end{cases}
        \implies\ \ \begin{cases}
            \vb{n}_{uv}=-k_v\vphi_u-k\vphi_{uv} \\
            \vb{n}_{vu}=-k_u\vphi_v-k\vphi_{vu}
        \end{cases}
    .\] So we must have \[
        k_v\vphi_u=k_u\vphi_v
    .\] \(\vphi_u,\vphi_v\) linearly independent \(\implies k_u=k_v=0\). So \(k\)
    is constant on the chart.

    Case 1: \(k=0\), then \(\vb{n}_u=\vb{n}_v=0\). Hence \(\vb{n}\) is a constant
    vector field. Then \[
        \left<\vphi(u,v),\vb{n}\right> 
    \] is constant since it has zero differentials. This shows \(S\) lie in a plane
    locally.

    Case 2: \(k\neq 0\), wlog assume \(k>0\), otherwise we can reverse the
    orientation. Then \[
        \begin{cases}
            \vphi_u=-\frac{1}{k}\vb{n}_u \\
             \vphi_v=-\frac{1}{k}\vb{n}_v
        \end{cases}
    .\] So \(\vphi+\frac{1}{k}\vb{n}\) is constant. (Note \(\vphi\) is the position
    vector). Let \(c=\vphi+\frac{1}{k}\vb{n}\). Then \[
        |\vphi-c|=\frac{1}{k}
    .\] This shows \(S\) lie in a sphere locally.

    So far we proved the result only in a local coordinate chart. Since the surface
    is connected and smooth, one can easily use a topology (covering) arguement
    to extend above result globally.
\end{proof}

\begin{remark}
    So far, we can characterize \(\mathbb{S}^2(R)\) in \(\mathbb{R}^3\) in the
    following two ways geometrically:
    \begin{enumerate}[(1)]
        \item \(\mathbb{S}^2(R)=\) the set of points having the same distance
            \(R\) to a fixed point \(c\). \ie\ \[
                |x-c|=R
            .\] 
        \item \(\mathbb{S}^2(R)=\) all points are umbilical. \ie\ \(\II=kI,k\neq 0\).
    \end{enumerate}
    Both of them are very useful in proving some surface is sphere.
\end{remark}
\begin{remark}
    \(H^2-K=\frac{1}{4}(k_1-k_2)^2\ge 0\). It vanishes at \(p\) when \(k_1=k_2\),
    \ie\ \(p\) umbilical. Together with Gauss-Bonnet thm, we have \[
        \int_{S}H^2\ge \int_{S}K=2\pi(2-2g)
    \] on a closed oriented surface, with equality holds iff \(S\) is sphere.
    In fact, we can get a stronger integral inequality.
\end{remark}
\begin{theorem}[Yau contest 2014]
    For a closed (oriented) surface \(S\subset \mathbb{R}^3\), \[
        \int_{S}H^2\ge 4\pi
    .\] Equality holds \(\iff S\) is a sphere.
\end{theorem}
\begin{proof}
    Recall the Gauss map \(\vb{n}\colon S\to \mathbb{S}^2(1)\), \[
        \dd{\vb{n}}\begin{bmatrix}
            \vphi_u \\ \vphi_v
        \end{bmatrix}=A\begin{bmatrix}
            \vphi_u \\ \vphi_v
        \end{bmatrix}
    .\] \[
        A=\II\cdot I^{-1}\implies K=\det A
    .\] For \(S\) closed surface, \(\vb{n}\) is surjective to \(\mathbb{S}^2(1)\).
    Hence \[
        \int_{S}H^2\ge \int_{S}|K|\ge \int_{\mathbb{S}^2}1=4\pi
    .\]

    (To see why \(\vb{n}\) is surjective, notice for any direction vector \(v_0\),
    there is a point \(p\in S\) \st\ \(\left<p,v_0\right> \) maximal. For this point
    one can prove the normal vector coincident with \(v_0\)).
\end{proof}

\begin{definition}
    \(W(S)=\int_{S}H^2\dd{\sigma}\) is called the Willmore energy of surface \(S\).
\end{definition}

Now we know \(W(S)\ge 4\pi\), it's natural to ask what's the exact lower bounded
of \(W(S)\) for given topology.

\begin{conjecture}[Willmore]
    Given any smooth ``immersed'' torus \(T\) in \(\mathbb{R}^3\), \[
        W(T)\ge 2\pi.
    \]
\end{conjecture}

This conjecture is settled in the case of ``embedded'' by Marques \& Neves (2014
Annals). They also showed the equality holds iff the torus is obtained by the
stereographic projection of the Clifford torus which is an ``embedded'' torus in
\(\mathbb{R}^4\). \[
    \frac{1}{\sqrt{2}}\mathbb{S}^1\times \frac{1}{\sqrt{2}}\mathbb{S}^1
    =\{\frac{1}{\sqrt{2}}(\cos\theta,\sin\theta,\cos\vphi,\sin\vphi):
    0\le \theta\le 2\pi,0\le \vphi\le 2\pi\}
.\] 

\begin{conjecture}[Lawson, proved by Brendle]
    Clifford torus is only minimally embedded torus in \(\mathbb{S}^3\).
    (\ie\ as a minimal surface in metric of \(\mathbb{S}^3\)).
\end{conjecture}

Finally, we introduce some other geometric concepts in term of the 2nd fundamental 
form.

\begin{definition}[Curvature lines]
    The curvature lines are integral curves of principal directions:
    \(\gamma(s)\subset S\) \st{} \(\gamma'(s)\) is principal direction. \ie\ \[
        \vb{n}'(s)=\dv{s}\vb{n}(\gamma(s))=-\lambda(s)\gamma'(s)
    .\] 
\end{definition}

\begin{remark}
    At an umbilical point, any normal section provides a principal direction, there
    are infinity many of them. However, if a point is not umbilical, \ie\ \(k_1\neq 
    k_2\), then we can find a local parametrization near \(p\), \st\ coordinate
    curves are curvature lines which are orthogonal to each other.
\end{remark}

\begin{exercise}
    Let \(\vphi(u,v)\) be a local parametrization, show that coordinate curves are
    curvature lines \(\iff F=f=0\).
\end{exercise}
\begin{comment}
\begin{proof}
    1. If \(\vphi(u,v_0)\) and \(\vphi(u_0,v)\) are curvature lines, \[
        \begin{cases}
            \dd{\vb{n}}(\vphi_u)=-\lambda\vphi_u \\
            \dd{\vb{n}}(\vphi_v)=-\mu\vphi_v
        \end{cases}
        \text{\ie}\quad\begin{cases}
            \vb{n}_u=-\lambda\vphi_u \\
            \vb{n}_v=-\mu\vphi_v
        \end{cases}
    .\] 
    So \(f=-\lambda F=-\mu F\). \ie\ \((\lambda-\mu)F=0\). If \(\lambda=\mu\),
    then the point is umbilical, it's enough to imply \(F=0\).
    If \(\lambda\neq \mu\), then \(F=0\implies f=0\).
\end{proof}
\end{comment}

Let's derive an equation of curvature lines.
If \(\alpha(t)=\vphi(u(t),v(t))\) is a curvature line, then \[
    \dd{\vb{n}}(\alpha'(t))=-\lambda(t)\alpha'(t)
.\] 



\newpage
