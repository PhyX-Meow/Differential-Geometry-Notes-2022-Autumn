\section{Rigidity of Sphere}
In this section, we present two rigidity results about spheres in \(\mathbb{R}^3\),
one is Liebmann's theorem, and another is Alexandorov's theorem.

\subsection{Liebmann's Theorem}
\begin{theorem}[Liebmann]
    Let \(S\) be a closed connected surface in \(\mathbb{R}^3\), assume \(S\) has
    constant Gaussian curvature, then \(S\) is a sphere.
\end{theorem}
\begin{proof}
    By compactness of \(S\), there must be an elliptic point on \(S\), so \(K>0\).
    
    \underline{Goal:} All points on \(S\) are umbilical.

    Then \(S\) must lie in the sphere \(\mathbb{S}(x_0,\frac{1}{K})\), it must be this
    sphere since it is closed and connected.

    To prove the goal, consider the function on \(S\) \[
        F=(k_1-k_2)^2
    \] where \(k_1,k_2\) are principal curvatures. If \(F\not\equiv 0\), assume
    \(F\) attain maximal at \(p\in S\), where \(F(p)>0\). Since \(K=k_1k_2>0\),
    wlog assume \(k_1(p)>k_2(p)>0\). Then in a neighbourhood of \(p\), we can assume
    \(k_1>k_2\). Note that \(F\) attains local maximum at \(p\), so \(k_1\) is
    attaining local maximum and \(k_2\) has local minimum.
    Choose local coordinate such that at \(p\),
    \begin{align*}
        I&=g_{11}(\dd{x^1})^2+g_{22}(\dd{x^2})^2 \\ 
        \II&=h_{11}(\dd{x^1})^2+h_{22}(\dd{x^2})^2
    .\end{align*}
    Let \(A_{11}=\frac{h_{11}}{g_{11}}=k_1,A_{22}=\frac{h_{22}}{g_{22}}=k_2\).
    In homework 10, we have computed the Codazzi equation \[
        \partial_2 \log\sqrt{g_{11}}=\frac{\partial_2 A_{11}}{A_{22}-A_{11}},
        \quad \partial_1\log\sqrt{g_{22}}=\frac{\partial_1 A_{22}}{A_{11}-A_{22}}
    .\] Hence \(\partial_2g_{11}=\partial_1g_{22}=0\) at \(p\).
    For Gaussian curvature,
    \begin{align*}
        K&=-\frac{1}{\sqrt{g_{11}g_{22}}}\left(\partial_1\left(\frac{\partial_1\sqrt{g_{22}}}{\sqrt{g_{11}}}\right)+\partial_2\left(\frac{\partial_2\sqrt{g_{11}}}{\sqrt{g_{22}}}\right)\right) \\
        \overset{\text{at }p}&{=}-\frac{1}{\sqrt{g_{11}g_{22}}}\left(\frac{\partial_1^2g_{22}}{2\sqrt{g_{11}g_{22}}}+\frac{\partial_2^2g_{11}}{2\sqrt{g_{11}g_{22}}}\right)
    .\end{align*}
    Note that \(k_1\) has local maximum \(\implies \partial_2^2k_1<0\). Hence \[
        \partial_2^2g_{11}\sim \frac{\partial_2^2k_1}{k_2-k_1}\ge 0
    .\] Similarly \[
        \partial_1^2g_{22}\sim \frac{\partial_1^2k_2}{k_1-k_2}\ge 0
    .\] Then \(K(p)\le 0\), this is a contradiction. Hence \(F\equiv 0\) on \(S\).
\end{proof}

Let's summarize a lemma we have shown in above proof.
\begin{lemma}
    Let \(S\) be a regular surface and \(p\in S\) such that
    \begin{itemize}
    \item \(K(p)>0\),
    \item \(p\) is local maximum of \(k_1\) and local minimum of \(k_2\).
    \end{itemize}
    Then \(p\) is an umbilical point of \(S\).
\end{lemma}

\begin{remark}
    With a similar argument we can show that:
\end{remark}
\begin{theorem}
    Let \(S\) be a connected closed surface, \(K>0\) everywhere. Assume \(H\) is
    constant, then \(S\) is sphere.
\end{theorem}
A key observation in the proof of both Liebmann's theorem and this one is that
one principal curvature is a decreasing function of the other.

\section{Another proof of Liebmann's theorem}
Our proof will be based on the following formula:
\begin{theorem}[Minkowski formula]
    Let \(S\) be a closed surface in \(\mathbb{R}^3\), \(p\in S\), then
    \begin{itemize}
    \item \(\int_{S}\dd{A}=-\int_{S}H\left<p,\vec{n}\right> \dd{A}\).
    \item \(\int_{S}H\dd{A}=-\int_{S}K\left<p,\vec{n}\right> \dd{A}\).
    \end{itemize}
\end{theorem}
Assume the formula first, we prove the Liebmann's theorem.
\begin{proof}
    Wlog assume the origin is contained inside \(S\), let \(p\) be the position vector
    field of \(S\) in \(\mathbb{R}^3\), \(\vec{n}\) be the inner normal. Since \(K\)
    is positive constant and \(H^2\ge K\). Choose a point \(q\) with maximum distance
    to the origin we see \(\II(q)>0\). Hence \(H\ge \sqrt{K}>0\) holds on \(S\) since
    \(H\) is continuous function. Then
    \begin{align*}
        -\int_{S}K\left<p,\vec{n}\right> \dd{A}&=\int_{S}H\dd{A}
        \ge \int_{S}\sqrt{K}\dd{A} \\
        &=\sqrt{K}\int_{S}\dd{A}=-\sqrt{K}\int_{S}H\left<p,\vec{n}\right> \dd{A} \\
        &\ge-K\int_{S}\left<p,\vec{n}\right> \dd{A}
    .\end{align*}
    Hence \(H^2=K\) and all points are umbilical.
\end{proof} 
Next we prove Minkowski's formula.
\begin{proof}
    Let \(\vphi(x^1,x^2)\) be local coordinate, in particular, \(p=\vphi(x^1,x^2)\)
    be the position vector, \(\vec{n}=\frac{\vphi_1\wedge \vphi_2}{|\vphi_1\wedge
    \vphi_2|}\) is the unit normal.

    To understand the integrand in the formula, let's do some perturbation. We perturb
    the surface along the normal direction, let \[
        \vphi_t(x^1,x^2)=\vphi(x^1,x^2)+t\vec{n}(x^1,x^2)
    .\] Then \[
        \dd{A_t}=\left|\pdv{\vphi_t}{x^1}\wedge \pdv{\vphi_t}{x^2}\right|
        \dd{x^1}\dd{x^2}
    .\] Consider the linearization 
    \begin{align*}
        \eval{\dv{t}}_{t=0}\dd{A_t}&=\frac{1}{|\partial_1\vphi\wedge \partial_2\vphi|}
        \left<\pdv{n}{x^1}\wedge \pdv{\vphi}{x^2}+\pdv{\vphi}{x^1}\wedge \pdv{n}{x^2},
        \pdv{\vphi}{x^1}\wedge \pdv{\vphi}{x^2}\right> \dd{x^1}\dd{x^2} \\
        &=\left<n_1\wedge \vphi_2+\vphi_1\wedge n_2,\vec{n}\right> \dd{x^1}\dd{x^2}
    .\end{align*}
    Note that \begin{gather*}
        n_1=-A_{11}\vphi_1-A_{12}\vphi_2 \\
        n_2=-A_{12}\vphi_1-A_{22}\vphi_2
    \end{gather*}
    Hence \begin{align*}
        n_1\wedge \vphi_2+\vphi_1\wedge n_2&=-A_{11}\vphi_1\wedge \vphi_2-A_{22}\vphi_1
        \wedge \vphi_2 \\ 
        &=-(A_{11}+A_{22})\vphi_1\wedge \vphi_2 \\ 
        &=-2H\vphi_1\wedge \vphi_2
    .\end{align*}
    So \begin{align*}
        \eval{\dv{t}}_{t=0}\dd{A_t}&=-2H\left<\vphi_1\wedge \vphi_2,\vec n\right> 
        =-2H|\vphi_1\wedge \vphi_2|\left<\vec{n},\vec{n}\right> \\
        &=-2H|\phi_1\wedge \vphi_2|\dd{x^1}\dd{x^2} \\
        &=-2H\dd{A}
    .\end{align*}
    Now we compute
    \begin{align*}
        -2H|\vphi_1\wedge \vphi_2|&=-2H\left<\vphi_1\wedge \vphi_2,n\right> \\
        &=\left<n_1\wedge \vphi_2+\vphi_1\wedge n_2,n\right>
        =\left<\vphi_1\wedge n_2,n\right> -\left<\vphi_2\wedge n_1,n\right> \\
        &=\pdv{x^1}\left<\vphi\wedge n_2,n\right> -\left<\vphi\wedge n_{21},n\right> 
        -\left<\vphi\wedge n_2,n_1\right> \\
        &\ -\pdv{x^2}\left<\vphi\wedge n_1,n\right> +\left<\vphi\wedge n_{12},n\right> 
        +\left<\vphi\wedge n_1,n_2\right> \\
        &=\pdv{x^1}\left<\vphi\wedge n_2,n\right> -\pdv{x^2}\left<\vphi\wedge n_1,
        n\right> +2\left<\vphi\wedge n_1,n_2\right> \\
        &=\pdv{x^1}\left<\vphi\wedge n_2,n\right> -\pdv{x^2}\left<\vphi\wedge n_1,
        n\right> +2\left<n_1\wedge n_2,\vphi\right> 
    .\end{align*}
    Note that \[
        n_1\wedge n_2=(A_{11}A_{22}-A_{12}^2)\vphi_1\wedge\vphi_2=K\vphi_1\wedge\vphi_2
    .\] Hence we have a local equality \[
        -2H|\vphi_1\wedge \vphi_2|=\pdv{x^1}\left<\vphi\wedge n_2,n\right> 
        -\pdv{x^2}\left<\vphi\wedge n_1,n\right> +2K|\vphi_1\wedge \vphi_2|
        \left<n,\vphi\right> 
    .\] To make it global one can integrate w.r.t. \(\dd{x^1}\wedge \dd{x^2}\),
    notice \[
        \left(\pdv{x^1}\left<\vphi\wedge n_2,n\right> -\pdv{x^1}\left<\vphi\wedge n_1,
        n\right> \right)\dd{x_1}\wedge \dd{x^2}
        =\dd\left(\left<\vphi\wedge n_2,n\right> \dd{x^2}+\left<\vphi\wedge n_1,
        n\right> \dd{x^2}\right)
    .\] Also \[
        |\vphi_1\wedge \vphi_2|\dd{x^1}\wedge \dd{x^2}
        =\sqrt{\det g}\dd{x^1}\dd{x^2}=\dd{A}
    .\] Hence \[
        -2\int_{S}H\dd{A}=2\int_{S}K\left<\vphi,n\right> \dd{A}
        +\int_{S}\dd(\cdots)
    .\] The third term is zero by Stokes' formula. We proved (2).
    
    For (1), again we start from \[
        n_1\wedge \vphi_2+\vphi_1\wedge n_2=-2H|\vphi_1\wedge \vphi_2|\vec{n}
    .\] Then \begin{align*}
        -2H\left<\vphi,\vec{n}\right> |\vphi_1\wedge \vphi_2|&=
        \left<n_1\wedge \vphi_2+\vphi_1\wedge n_2,\vphi\right> \\
        &=\left<n_1\wedge \vphi_2,\vphi\right> +\left<\vphi_1\wedge n_2,\vphi\right> \\
        &=\pdv{x^1}\left<n\wedge \vphi_2,\vphi\right> -\left<n_1\wedge \vphi_{21},
        \vphi\right> -\left<n\wedge \vphi_2,\vphi_1\right> \\
        &\ -\pdv{x^2}\left<n\wedge \vphi_1,\vphi\right> +\left<n\wedge \vphi_{12},
        \vphi\right> +\left<n\wedge \vphi_1,\vphi_2\right> \\
        &=\pdv{x^1}\left<n\wedge \vphi_2,\vphi\right> -\pdv{x^2}\left<n\wedge \vphi_1,
        \vphi\right> +2|\vphi_1\wedge \vphi_2|
    .\end{align*}
    Hence similarly, \[
        -2H\left<\vphi,n\right> \dd{A}=2\dd{A}+\dd(\cdots )
    .\] And then \[
        \int_{S}\dd{A}=-\int_{S}H\left<\vphi,n\right> \dd{A}
    \] by stokes' formula.
\end{proof}

\subsection{A little generalization}
Now we provide another proof of Liebmann's theorem by using a more general argument.
We started with a very useful and important formula which is frequently used in the
study of hypersurface \(M^n\) in \(\mathbb{R}^{n+1}\).

Let \(\dd{s}^2=g_{ij}\dd{x^i}\dd{x^j}\), \(\II=h_{ij}\dd{x^1}\dd{x^j}\) be the 1st
and 2nd fundamental form. And \(\dd{n}\colon TM\to TM\) be given by \[
    \begin{pmatrix}
        n_1 \\ \vdots \\ n_n
    \end{pmatrix}=-\begin{pmatrix}
    \\
    \ & A_i^j & \ \\
    \ 
    \end{pmatrix}_{n\times n}\begin{pmatrix}
        \vphi_1 \\ \vdots \\ \vphi_n
    \end{pmatrix},\qquad A_i^j=h_{ik}g^{kj}
.\] The length of the 2nd fundamental form is defined by \[
    |\II|^2=h_{ij}h_{kl}g^{ik}g^{jl}=A_j^k A_k^j=\tr A^2
.\] We compute
\begin{align*}
    \frac{1}{2}\triangle |\II|^2&=\frac{1}{2}g^{pq}\nabla_p\nabla_q(g^{ik}g^{jl}h_{ij}
    h_{kl}) \\ &=\frac{1}{2}g^{pq}g^{ik}g^{jl}\left(\nabla_p\nabla_q h_{ij}h_{kl}
    +\nabla_p\nabla_q h_{kl}h_{ij}+\nabla_p h_{ij}\nabla_q h_{kl}+\nabla_p h_{kl}
    \nabla_q h_{ij}\right) \\
    &=g^{pq}g^{ik}g^{jl}\left(\nabla_p\nabla_q h_{ij}h_{kl}+\nabla_p h_{ij}\nabla_q
    h_{kl}\right) \\
    &=|\nabla\II|^2+g^{pq}g^{ik}g^{jl}(\nabla_p\nabla_q h_{ij})h_{kl}
.\end{align*}
In the following computation, we first choose local orthonormal frame \(\{e_1,\ldots,
e_n\}\). Note that \(\triangle |\II|^2\) is a global expression, which is independent
of choice of coordinate. We can always choose suitable coordinate frame to simplify
the computation. Note that \(g_{ij}=\delta_{ij}\) now, and we do not distinguish
upper and lower indices.
\begin{align*}
    \frac{1}{2}\triangle |\II|^2&=|\nabla\II|^2+\sum_{p,i,j}(\nabla_p\nabla_p h_{ij})
    h_{ij} \\ &=|\nabla\II|^2+\sum_{p,i,j}(\nabla_p\nabla_i h_{pj})h_{ij}
    \quad \text{(By Codazzi equation)} \\
    &=|\nabla\II|^2+\sum_{p,i,j}(\nabla_i\nabla_p h_{pj}+R_{pipk}h_{kj}+R_{pijk}h_{pk})
    h_{ij} \quad \text{(Ricci identity)} \\
    &=|\nabla\II|^2+\sum_{p,i,j}(\nabla_i\nabla_j h_{pp}+R_{pipk}h_{kj}-R_{pikj}h_{pk})
    h_{ij} \\
    &=|\nabla\II|^2+\sum_{i,j}(n\nabla_i\nabla_j H)h_{ij}
    +\sum_{p,i,j,k}(R_{pipk}h_{kj}-R_{pikj}h_{pk})h_{ij}
.\end{align*}
Note we still have freedom to choose local frame such that \[
    h_{ij}=\lambda_i\delta_{ij}
.\] Then 
\begin{align*}
    &\ \sum_{p,i,j,k}(R_{pipk}h_{kj}-R_{pikj}h_{pk})h_{ij}
    &=\sum R_{pipi}\lambda_i\lambda_i-R_{pipi}\lambda_p\lambda_i \\ 
    &=\sum \lambda_i(\lambda_i-\lambda_p)R_{pipi} \\
    &=\sum(\lambda_i-\lambda_p)^2R_{pipi}-\sum\lambda_i(\lambda_i-\lambda_p)R_{ipip} \\
    &=\sum(\lambda_i-\lambda_p)^2R_{pipi}-\sum\lambda_i(\lambda_i-\lambda_p)R_{ipip}
.\end{align*}
Hence we got \[
    \frac{1}{2}\triangle|\II|^2=|\nabla\II|^2+n\sum(\nabla_i\nabla_j H)h_{ij}
    +\frac{1}{2}\sum (\lambda_i-\lambda_p)^2R_{ipip}
.\] 
\begin{remark}
    In above process of deriving the equality. We only used the assumption that
    \(M\hookrightarrow \mathbb{R}^{n+1}\) is a hypersurface. Moreover, the existence
    of local orthonormal frame \(\{e_1,\ldots,e_n,e_{n+1}\}\) with \(e_{n+1}\) be the
    unit normal, is just analog to the case of surfaces in \(\mathbb{R}^3\). We
    summarize this important formula as a proposition.
\end{remark}

\begin{prop}
    Let \((M^n,g)\hookrightarrow\mathbb{R}^{n+1}\) be a smooth hypersurface.
    \(\II=h_{ij}\dd{x^i}\dd{x^j}\) be the 2nd fundamental form. In local orthonormal
    frame, let \(h_{ij}=\lambda_i\delta_{ij}\), then \[
        \frac{1}{2}\triangle|\II|^2=|\nabla\II|^2+n\sum(\nabla_i\nabla_j H)h_{ij}
        +\frac{1}{2}\sum (\lambda_i-\lambda_p)^2R_{ipip}
    .\] 
\end{prop}
We can further handle this formula:
\begin{align*}
    \frac{1}{2}\triangle|\II|^2&=|\nabla\II|^2+n\sum\nabla_i(\nabla_j H h_{ij})
    -n\sum\nabla_j H\nabla_i h_{ij}+\frac{1}{2}\sum (\lambda_i-\lambda_p)^2R_{ipip} \\
    &=|\nabla\II|^2+\underbrace{n\sum\nabla_i(\nabla_j H h_{ij})}_{\text{divergence
    form}}-n^2|\nabla H|^2+\frac{1}{2}\sum(\lambda_i-\lambda_p)^2R_{ipip}
.\end{align*}

\begin{proof}[One more proof of Liebmann's theorem]
    In terms of the orthonormal frame we have chosen, we have \[
        |\II|^2=\lambda_1^2+\lambda_2^2,\quad K=\lambda_1\lambda_2,\quad
        H=\frac{1}{2}(\lambda_1+\lambda_2)
    .\] Then \[
        2K=4H^2-|\II|^2\qquad\text{(1)}
    .\] If \(K=C>0\) then \(R_{1212}=K>0\), so \(\frac{1}{2}(\lambda_1-\lambda_2)^2
    R_{1212}\ge 0\). Take derivative of (1), we have \[
        0=2\nabla_i K=8H\nabla_i H-2h_{pq}\nabla_i h_{pq}
    .\] Hence \[
        4H\nabla_i H=h_{pq}\nabla_i h_{pq}\leftarrow\text{this is a vector}
    .\] Take length on both sides, we get \[
        4|H||\nabla H|=|h_{pq}\nabla_i h_{pq}|\le |\II||\nabla\II|\quad
        \text{(Cauchy-Schwarz)}
    .\] By (1), \(K\ge 0\implies |\II|\le 2|H|\), hence \(|\nabla\II|\ge 2|\nabla H|\).
    Put this into the formula, we have \[
        \frac{1}{2}\triangle |\II|^2=\underbrace{|\nabla\II|^2-4|\nabla H|^2}_{\ge 0}
        +2\sum\nabla_i(\nabla_j H h_{ij})+\frac{1}{2}(\lambda_1-\lambda_2)^2K
    .\] Integrate over \(S\), we have \[
        0\ge \frac{K}{2}\int_{S}(\lambda_1-\lambda_2)^2\dd{A}
    .\] Hence \(\lambda_1=\lambda_2\) everywhere since \(K>0\).
\end{proof}
\begin{remark}
    With a similar argument, in Riemannian Geometry we can show
    \begin{enumerate}[(1)]
    \item \(M^n\hookrightarrow\mathbb{R}^{n+1}\) closed hypersurface, assume \(H\)
        is constant and \(\mathrm{Ric}(M)\ge 0\), then \(M\cong \mathbb{S}^n\).
    \item \(M^n\hookrightarrow\mathbb{R}^{n+1}\) closed hypersurface, assume the
        scalar curvature \(S\) is constant and \(\mathrm{Ric}\ge 0\), then
        \(M\cong \mathbb{S}^n\).
    \end{enumerate}
    Reference: 1. Do Carmo's book. 2. Peng Jiagui, Chen Qing's book.
    3. Shen Yibing's book, chaper 5 section 3.
\end{remark}

\subsection{Hadamard-Stoker theorem}
\begin{theorem}
    If \(S\) is a connected closed surface in \(\mathbb{R}^3\), with Gaussian
    curvature \(K>0\). Then \(S\) is diffeomorphic to \(\mathbb{S}^2\).
\end{theorem}
\begin{proof}
    Consider the Gaussian map \(N\colon S\to \mathbb{S}^2\). First, since \(K_p=
    \det N_p>0\), \(\dd{N_p}\) is a linear iso. We know \(N\) is a local diffeomorphism
    near each \(p\in S\). Further more, \(N(S)\) is an open subset of \(\mathbb{S}^2\).
    On the other hand, \(S\) is compact, \(N\) is continuous. So \(N(S)\) is a
    compact subset of \(\mathbb{S}^2\). Hence \(N(S)\) must be whole \(\mathbb{S}^2\),
    and \(N\) is a surjection.
    
    Next, we show \(N\) is injective. If \(\exists\,p\neq q\) on \(S\) such that
    \(N(p)=N(q)\), using the Hausdorff property we can find neighbourhood
    \(U\) of \(p\), \(V\) of \(q\) such that \(U\cap V=\emptyset\). But \(N(U)=
    N(V)=W\) is neighbourhood of \(N(p)=N(q)\). Hence \(N\colon S\setminus U\to
    \mathbb{S}^2\) is still surjective.

    On one hand, \[
        \int_{S\setminus U}K\dd{A}\ge \int_{\mathbb{S}^2}\dd{A_{\mathbb{S}^2}}
        =4\pi
    .\] On the other hand since \(K>0\), \(\int_{S}K>0\), by Gasuss-Bonnet theorem, \[
        \int_{S}K=4\pi
    .\] Hence \[
        4\pi=\int_{S}K=\int_{S\setminus U}K+\int_{U}K>\int_{S\setminus U}K\ge 4\pi
    .\] Contradiction. Thus \(N\) is surjective.
\end{proof}
Reference: Motiel-Ros's book or Peng Jiagui, ChenQing's book.
