\section{Linear algebra convention and its geometric explanation}
\begin{itemize}
    \item We use ``ROW VECTOR'' in this course, \ie\
    \[v\in \mathbb{R}^n, v=(v_1,v_2,\cdots,v_n)\]
    \item let $e_1=(1,0,\cdots,0),\cdots,e_n=(0,\cdots,1)$ be the standard basis, then 
    \[v=\sum_{i=1}^nv^i e_i=
    \begin{bmatrix}
        v^1& v^2& \cdots & v^n
    \end{bmatrix}
    \begin{bmatrix}
        e_1\\
        e_2\\
        \vdots\\
        e_n
    \end{bmatrix}
    \]
    \item $\varphi\colon \mathbb{R}^n\to \mathbb{R}^n$ (non-degenerate) linear map
    \[v\mapsto \varphi(v)=v\cdot A.\]
    This corresponds to the right action of $GL(n,\mathbb{R})$ on $\mathbb{R}^n$.
    \[\Rightarrow \varphi(e_i)=e_j \cdot A=\sum_{i=1}^n A\indices{_j^i}e_i\text{ (taking the j-th row of }A\text{)}\]
    
    \[A\indices{_j^i}\begin{cases}
        \text{upper index: column index}\\
        \text{lower index: row index}
    \end{cases}\]
    \[\Rightarrow \varphi\begin{bmatrix}
        e_1\\
        e_2\\
        \vdots\\
        e_n
    \end{bmatrix}=\begin{bmatrix}
       \varphi( e_1)\\
        \varphi (e_2)\\
        \vdots\\
        \varphi(e_n)
    \end{bmatrix}=\begin{bmatrix}
        e_1\cdot A\\
        e_2\cdot A\\
        \vdots\\
        e_n\cdot A
    \end{bmatrix}=
    A \begin{bmatrix}
        e_1\\
        e_2\\
        \vdots\\
        e_n
    \end{bmatrix}\]
\end{itemize}
\begin{remark}[Important!]
    In row vector convention, a non-degenerate linear map corresponds to the right action of $GL(n,\mathbb{R})$ on $\mathbb{R}^n$. But this induces left action of $GL(n,\mathbb{R})$ on the orthonormal basis (frame) $\{e_1,e_2,\ldots,e_n\}$. This phenomenon provides an important example in differential geometry, which will be explained later in the theory of principle bundle.(\ie\ let $G$ be a lie group, $G\curvearrowright M$ being a right action, where $M$ is a differentiable manifold, then this right action induces a left action of $G$ on the frame bundle of $M$.)
 \end{remark}
 
 
 Let $\{\tilde{e}_1,\ldots,\tilde{e}_n\}$ be another basis of $\mathbb{R}^n$. Let $f$ be the corresponding linear map, \ie\
 \[f\begin{bmatrix}
    e_1\\
    e_2\\
    \vdots\\
    e_n
\end{bmatrix}=\begin{bmatrix}
    \tilde{e}_1\\
    \tilde{e}_2\\
    \vdots\\
    \tilde{e}_n
\end{bmatrix}=B \cdot \begin{bmatrix}
    e_1\\
    e_2\\
    \vdots\\
    e_n
\end{bmatrix}\]
\[\Rightarrow \tilde{e}_k=\sum_{j=1}^n B\indices{_k^j}e_j\]
We compare the matrix of $\varphi$ in terms of $\left\{\tilde{e}_1 \cdots \tilde{e}_n\right\}$
\[
    \varphi\left[\begin{array}{c}\tilde{e}_1 \\ \vdots \\ \tilde{e}_n\end{array}\right]=\varphi\left[B\left[\begin{array}{c}\tilde{e}_1 \\ \vdots \\ e_n\end{array}\right]\right]=B \cdot \varphi\left[\begin{array}{c}e_1 \\ \vdots \\ e_n\end{array}\right]\text{(linearity of }\varphi\text{)}
\]
\[
    =B A\left[\begin{array}{c}
    e_1 \\
    \vdots \\
    e_n
    \end{array}\right]=B A B^{-1}\left[\begin{array}{c}
    \tilde{e}_1 \\
    \vdots \\
    e_n
    \end{array}\right]
\]
Note in this case.
\[
(\varphi \circ f)\left[\begin{array}{c}
e_1 \\
\vdots \\
e_n
\end{array}\right]=B A\left[\begin{array}{c}
e_1 \\
\vdots \\
e_n
\end{array}\right]
\]
In terms of entries,
\begin{align*}
    \varphi(\tilde{e}_k) &=\varphi(\sum_{j=1}^n B\indices{_k^j} e_j)=\sum_{j=1}^n B\indices{_k^j} \varphi(e_j) \quad \text { (linearity) } \\
    &=\sum_{i, j=1}^n B\indices{_k^j} A\indices{_j^i} e_i=\sum_{i,j,p=1}^n B\indices{_k^j} A\indices{_j^i}(B^{-1})\indices{_i^p} \widetilde{e}_p
\end{align*}
\begin{remark}
    This computation tells that the row vector convention yields to the fact that $GL(n,\mathbb{R})$ acting on itself from the right when we consider the action of $GL(n, \mathbb{R})$ on $\mathbb{R}^n$.
    In modern Geometry, it's more common to use column vector as convention. This row vector convention was adopted by S.S. Chern and also Do Cormo's book.
\end{remark}
\section{Parametrized Curves}
\begin{definition}
    Let $I=(a,b)$, if $\alpha\colon I\to \mathbb{R}^3$ is a $C^\infty$ map,
    \[t \mapsto (x(t),y(t),z(t))\]
    then $\alpha(t)$ is a parametrized differentiable curve in $\mathbb{R}^3$. The image of $\alpha$ is called the trace of the curve. 
\end{definition}
\begin{remark}
    \hfill
    \begin{enumerate}[1)]
        \item $a,b$ could be finite number or infinity.
        \item Same curve may have different parametrizations.
        \item The parametrization automatically gives the direction of the motion on the curve.
        \item ``Differentiable'' just means $\alpha(t)$ is a $C^\infty$ \textbf{map}, it does not say the (trace of) curve can not have singularities.
    \end{enumerate}
\end{remark}
\begin{example}
    \hfill
    \begin{enumerate}[(1)]
        \item $\alpha(t)=(t,|t|)$ is not a differentiable curve.
        \begin{center}
            \begin{tikzpicture}
                \draw[domain=-2:0,smooth,variable=\t,black]
                plot (\t,{-\t});
                \draw[domain=0:2,smooth,variable=\t,black]
                plot (\t,\t);    
                \draw (0,0) node{$\bullet$};
                \draw[->] (-2,0) -- (2,0) node[right] {$x$};
                \draw[->] (0,-0.5) -- (0,2.5) node[above] {$y$};
            \end{tikzpicture}
        \end{center}
        \item $\alpha=(t^3,t^2)$ is a differentiable curve. It can be also given by a equation $y^3=x^2$, which is a cuspidal cubic curve.
        \begin{center}
            \begin{tikzpicture}
                \draw[domain=-1.5:1.5,smooth,variable=\t,black]
                plot ({\t^3},{\t*\t});
                \draw (0,0) node{$\bullet$};
                \draw[->] (-2,0) -- (2,0) node[right] {$x$};
                \draw[->] (0,-0.5) -- (0,2.5) node[above] {$y$};
            \end{tikzpicture}
        \end{center}
        \item $\alpha(t)=(t^2-1,t^3-t)$. This parametrization appers in the ``blow-up'' process of $y^2=x^3+x^2$. Here ``blow-up'' is introducing tangents to seperate points.
        \begin{center}
            \begin{tikzpicture}
                \draw[domain=-1.5:1.5,smooth,variable=\t,black]
                plot ({\t*\t-1},{\t^3-\t});
                \draw (0,0) node{$\bullet$};
                \draw[->] (-2,0) -- (2,0) node[right] {$x$};
                \draw[->] (0,-2.5) -- (0,2.5) node[above] {$y$};
            \end{tikzpicture}
        \end{center}
    \end{enumerate}
\end{example}
\begin{remark}
    (2) and (3) above may be the first examples you'll see in an algebraic geometry course.
\end{remark}
\noindent
\textbf{Question}: At the origin, what can you obsefve on (2) and (3)?

\noindent
\textbf{Answer}:
    (2) $\alpha'(0)=0$.
    (3) $\alpha$ is not one to one, but $\alpha'(0)\neq 0$.

\noindent 
\textbf{Question}: Define a differentiable curve in $\mathbb{R}^3$ and $\mathbb{S}^n$.
\begin{remark}
    Among above differentiable parametrizations, (2) and (3) are differentiable curves. However, if we take $\beta(t)=(t,t^{\frac{2}{3}})$, this also parametrizes (2), but it's not a differentiable curve!
\end{remark}
\begin{definition}
    Let $\alpha(t)\colon I\to \mathbb{R}^3$ be a parametrized differentiable curve, then at $t_0\in I$.
    \[ \alpha'(t_0)=(x'(t_0),y'(t_0),z'(t_0))\]
    is the velocity of $\alpha(t)$ at $t_0$.
    \begin{enumerate}[(1)]
        \item If $\alpha'(t_0)\neq 0$, we call $\alpha(t_0)$ a regular point.
        \item If $\alpha'(t_0) = 0$, we call $\alpha(t_0)$ a singular point.
        \item If for all $t\in I$, $\alpha'(t)\neq 0$, we call $\alpha(t)$ a regular curve.
    \end{enumerate}
\end{definition}
\noindent
\textbf{Question}: What can you say about $C^\infty$ parametrization for a regular curve?
\begin{quotation}
Regular curve $\Longleftrightarrow $ at each point, there is a unique tangent line.
\end{quotation}
\begin{example}
$\alpha(t)=(t^3,t^2)$ is not a regular curve. (Since $\alpha'(0)=0$)
\begin{center}
    \begin{tikzpicture}
        \draw[domain=-1.5:1.5,smooth,variable=\t,black]
        plot ({\t^3},{\t*\t});
        \draw (0,0) node{$\bullet$};
        \draw[->] (-2,0) -- (2,0) node[right] {$x$};
        \draw[->] (0,-0.5) -- (0,2.5) node[above] {$y$};
    \end{tikzpicture}
\end{center}
\end{example}
\begin{example}
$\alpha(t)=(t^2-1,t^3,t)$ is a regular curve.
\begin{center}
    \begin{tikzpicture}
        \draw[domain=-1.5:1.5,smooth,variable=\t,black]
        plot ({\t*\t-1},{\t^3-\t});
        \draw (0,0) node{$\bullet$};
        \draw[->] (-2,0) -- (2,0) node[right] {$x$};
        \draw[->] (0,-2.5) -- (0,2.5) node[above] {$y$};
    \end{tikzpicture}
\end{center}
\end{example}
\begin{definition}
Let $\alpha(t)$ be a regular curve, then the tangent line at $t_0$ is \[l(t)=\alpha(t_0)+\alpha'(t_0)(t-t_0))\]
\begin{center}
    \begin{tikzpicture}[scale=1.25]
    \draw (0, -1) .. controls (1.5, 2) and (2, -3) .. (4,1.5)    % 绘制曲线
        node[
            pos = 0.6,    % 设置切点在曲线上的位置
            sloped,    % 设置node按曲线斜率旋转
            anchor = south west
            ] (N) {};
         

    \draw[blue]($(N.south west)!0.6cm!180:(N.south east)$) -- ($(N.south west)!0.6cm!(N.south east)$);
    \draw (0, -1) .. controls (1.5, 2) and (2, -3) .. (4,1.5)
        node[
        pos = 0.2,    % 设置切点在曲线上的位置
        sloped,    % 设置node按曲线斜率旋转
        anchor = south west
        ] (M) {}; 
    \draw[cyan]($(M.south west)!0.6cm!180:(M.south east)$) -- ($(M.south west)!0.6cm!(M.south east)$);
    \draw (0, -1) .. controls (1.5, 2) and (2, -3) .. (4,1.5)
        node[
        pos = 0.9,    % 设置切点在曲线上的位置
        sloped,    % 设置node按曲线斜率旋转
        anchor = south west
        ] (M) {}; 
    \draw($(M.south west)!1cm!180:(M.south east)$) -- ($(M.south west)!1cm!(M.south east)$);
    \end{tikzpicture}
\end{center}
\end{definition}
\begin{definition}
    Let $\alpha(t)$ be a regular curve, the arc-length of $\alpha(t)$ is 
    \[s(t)=\int_{t_0}^t \left|\alpha'(t)\right|\dd t.\]
    Then $s'(t)=\left|\alpha'(t)\right|$
\end{definition}
\textbf{Question} What's $\left|\alpha'(t)\right|$?

 
$\alpha(t)\colon I\to \mathbb{R}^3$ is a curve in $\mathbb{R}^3$. Here on $\mathbb{R}^3$, as the Euclidean space, we always assume the standard Euclidean inner product on it, \ie\ $\forall u=(u_1,u_2,u_3),v=(v_1,v_2,v_3)$
\[\langle u,v\rangle=u_1 v_1+u_2 v_2+ u_3 v_3=\sum_{i,j=1}^3\delta_{ij}u_i v_j\]
Let $\alpha(t)=(x(t),y(t),z(t)),\alpha'(t)=(x'(t),y'(t),z'(t))$, then $\left|\alpha'(t)\right|=\sqrt{\langle\alpha'(t),\alpha'(t)\rangle}$
\begin{exercise}
    Review vector Calculations, such as dot product, cross product and their properties, especially geometric meanie of these calculation, such as length, area, volume, angle, orientation, etc.
\end{exercise}
\noindent
\textbf{Question}: Can you define the arclength of a regular curve in $\mathbb{R}^n$? How about on $\mathbb{S}^n$? \\
$\bullet$ Arclength parameter(an intrinsic parametrization of a curve)
\begin{example}
On a straight line, x=t describes the distance of the point away from the origin.
\begin{center}
    \begin{tikzpicture}
        \draw[->] (-2,0) -- (2,0) node[right] {};
        \draw (-1,0.05) node{$\bullet$};
        \draw (0.2,0.2) node{$t$};
        \draw[->,blue] (-1,0.1)--(0,0.1) node[right] {};
    \end{tikzpicture}
\end{center}
\end{example}
On a general curve, we also want ``some'' parameter, which describes the arclength of point away from the initial point. This can happen iff $|\alpha'(t)|=1$,\ie\ a point on the curve moves in a unit speed.
\[\Rightarrow s(t)=\int_0^t \dd t=t.\]
\textbf{Question}: For a given regular curve $\alpha(t)\colon I\to \mathbb{R}^3$, how to find such parameter?

\noindent 
\textbf{Answer}: $s(t)=\int_{t_0}^t \left|\alpha'(t)\right|\dd t$ is a function in t, and $s'(t)=\left|\alpha'(t)\right|\neq 0$(because the curve is regular). By the implicit function theorem, there is a function 
\[
    t=t(s),t'(s)=\frac{1}{\left|\alpha'(t)\right|}.
\]
This implies that 
\[
    \alpha(t)=\alpha(t(s))=(x(t(s)),y(t(s)),z(t(s)))
\]
\[
    \left|\alpha'(s)\right|=\left|\alpha'(t)t'(s)\right|=\left|\alpha'(t)\right|\left|t'(s)\right|=1
\]

\boxed{\textbf{Convention}} In this course, we only consider differentiable regular curves, which are parametrized by the arclength (for convenience).
\begin{remark}
    In this course, we only consider the curve without self-intersecting points, i.e curves ``embedded into'' $\mathbb{R}^3$. Here ``embedded'' means $d\alpha$ is a linear isomorphism and $\alpha$ is homeomorphic to its image.
\end{remark}
