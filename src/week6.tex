\section{Orientation of a regular surface}
\begin{question}
    Consider \(\mathbb{S}^2\) with spherical parametrization, are following two
    parametrizations the same?
    \begin{itemize}
        \item \((\vphi,\theta)\longmapsto(\sin\vphi\cos\theta,\sin\vphi\sin\theta,
            \cos\vphi)\)
        \item \((\theta,\vphi)\longmapsto(\sin\vphi\cos\theta,\sin\vphi\sin\theta,
            \cos\vphi)\)
    \end{itemize}
\end{question}

The answer is NO\@! They give different normal directions which decide ``inside'' and
``outside'' of \(\mathbb{S}^2\).

\noindent\underline{\bfseries Recall:} \(\mathbb{R}^2\) is orientable. It has two
orientations. Given a basis \(\{e_1,e_2\}\), let
\begin{align*}
    J_{+}=\Bigl\{E_{+}=\{e_1^+,e_2^+\}&:E_{+}\text{ has same orientation as }E.\\
    &\text{ \ie\ }\begin{bmatrix}
        e_1^+\\ e_2^+
    \end{bmatrix}=\underbrace{\begin{bmatrix}
        a_{11} & a_{12} \\
        a_{21} & a_{22}
    \end{bmatrix}}_{A_{+}}\begin{bmatrix}
        e_1 \\ e_2
    \end{bmatrix}
    \text{ and }\det A_{+}>0\Bigr\};
\end{align*}
\begin{align*}
    J_{-}=\Bigl\{E_{-}=\{e_1^-,e_2^-\}&:E_{-}\text{ has same orientation as }E.\\
    &\text{ \ie\ }\begin{bmatrix}
        e_1^-\\ e_2^-
    \end{bmatrix}
    =A_{-}\begin{bmatrix}
        e_1 \\ e_2
    \end{bmatrix}
    \text{ and }\det A_{-}>0\Bigr\}.
\end{align*}
Then either \(J_{+}\) or \(J_{-}\) gives an orientation of \(\mathbb{R}^2\).

\begin{question}\hfill
\begin{enumerate}[(1)]
    \item Let \(S\) be a regular surface, how can we define orientation?
    \item Are all surfaces orientable?
\end{enumerate}
\end{question}

Assume we have two parametrizations around \(p\):
\begin{align*}
    F&\colon U\to S &&T_p S\cong \mathbb{R}^2=\Span\{F_u,F_v\} \\
    G&\colon V\to S &&T_p S\cong \mathbb{R}^2=\Span\{G_\alpha,G_\beta\}
.\end{align*}
Note near \(p\), \(F(u,v)=G(\alpha,\beta)\implies \) \[
    \begin{bmatrix}
        G_\alpha \\ G_\beta
    \end{bmatrix}=\begin{bmatrix}
        u_\alpha & v_\alpha \\
        u_\beta & v_\beta
    \end{bmatrix}\begin{bmatrix}
        F_u \\ F_v
    \end{bmatrix}
.\] 
\ie\ Two parametrizations \(F\) and \(G\) give the same orientation iff \[
    \left|\pdv{(u,v)}{(\alpha,\beta)}\right|>0
.\] 

\begin{definition}[Orientation of a regular surface]
    We say a regular surface \(S\) is orientable if there exists a coordinate
    chart covering of \(S\), \st\ if a point \(p\) belongs to two charts, then the
    induced basis on \(T_p S\) by the two parametrizations have the same orientation
    in above sense.
\end{definition}

\begin{remark}\hfill
\begin{enumerate}[(1)]
    \item ``Orientation'' is a global intrinsic property. ``Orientability'' is
    essentially reflected by the topology of the surface. In algebraic topology
    course, you'll see that the orientability is determined by the 1st
    Stiefel-Whitney class in \(H^1(M,\mathbb{Z}/2\mathbb{Z})\) for a vector
    bundle on topological manifold \(M\).
    With additional ``smooth'' structure on \(M\), we have more ways to check
    the orientability, for example, \(\exists\) a nowhere vanishing top
    differential form (or say, a volume form).
    \item One of important applications of ``orientation'' in this course (and
    later in Reimann Geometry) is allowing us to define ``integration'' on a regular
    surface.
\end{enumerate}
\end{remark}

\begin{exercise}
    Give a definition of orientation of \(\mathbb{R}^n\) as a vector space.
\end{exercise}
