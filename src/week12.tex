\subsection{Geodesic curvature}
Let \(\alpha(s)\colon I\to S\) be a regular curve, at each point \(p\in \alpha\).
We have two o.n. frames:
\begin{enumerate}[(1)]
    \item Frenet's formula \(\{\vec{t}(s),\vec{n}(s),\vec{b}(s)\}\);
    \item \(\{\vec t(s),N(s)\times \vec t(s),N(s)\}\), \(N(s)\) is unit normal of
        \(S\) at \(\alpha(s)\).
\end{enumerate}
Using the second frame, we have \[
    \left<\alpha''(s),N(s)\right> =\II(\alpha'(s),\alpha'(s))=k_n,
    \text{ normal curvature at }\alpha(s)
.\] 

\begin{definition}[Geodesic curvature]
    \[
        k_g:=\left<\alpha''(s),N(s)\times \vec t(s)\right> 
    \] is called geodesic curvature.
\end{definition}

By the Frenet formula, \(\alpha''(s)=\vec t'(s)=k\vec n(s)\), let \(\theta\) be the
angle between \(\vec n(s)\) and \(N(s)\), then
\begin{align}
    k_n&= k\cos\theta=\left<\alpha''(s),N(s)\right> &\text{normal curvature} \\
    k_g&= \pm k\sin\theta=\left<\alpha''(s),\right> &\text{geodesic curvature}
.\end{align}
The sign of expression of \(k_g\) depends on the choice of normal of surface
and orientation of \(\alpha(s)\).

Hence \[
    k^2=k_n^2+k_g^2
.\] And \(\alpha(s)\) is geodesic \(\iff \vec n\parallel N\iff k_n=k\iff k_g=0\).

\begin{exercise}[Liouvill formula]
    Let \(\vphi(u,v)\) be an orthogonal parametrization of a regular surface \(S\),
    \(\alpha(s)\) be a regular curve on \(S\) with arclength parameter.
    Let \(\theta(s)\) be the angle between \(u\)-curve and \(\alpha'(s)\). Then \[
        k_g=\dv{\theta}{s}-\frac{1}{2\sqrt{G}}\pdv{\log E}{v}\cos\theta
        +\frac{1}{2\sqrt{E}}\pdv{\log G}{u}\sin\theta
    .\] 
\end{exercise}

\subsection{Length minimizing curves (variational viewpoint)}
\((M,g)\) be Riemannian manifold, let \(\gamma(s)\colon [0,l]\to M\) be a curve
with arc-length parameter. \(\gamma(0)=:p,\gamma(l)=:q\). Let \(\Gamma(s,t)\) be a
family of smooth variation with endpoints fixed.
\ie\ \(\Gamma(s,t)\colon [0,l]\times (-\eps,\eps)\to M\) \st\ \(\forall\,t\in (-\eps,
\eps)\), \(\Gamma(s,t_0)\) is a \(C^\infty\) curve with initial point \(p\) and
endpoint \(q\), and, \(\Gamma(s,0)=\gamma(s)\).

Consider the arc-length functional \[
    L(\Gamma(s,t))=\int_{0}^{l}\left|\pdv{\Gamma(s,t)}{s}\right|\dd{s}
.\] 
Then
\begin{equation}\label{eq:variation-geodesic}
\begin{split}
    \eval{\dv{t}}_{t=0}L(\Gamma(s,t))
    &=\int_{0}^{l}\frac{1}{2}\left<\dot\gamma(s),\dot\gamma(s)\right> ^{-\frac{1}{2}}
    \eval{\dv{t}}_{t=0}\left<\dv{\Gamma(s,t)}{s},\dv{\Gamma(s,t)}{s}\right> \dd{s} \\
    &=\int_{0}^{l}\left<\eval{\frac{\mathrm{D}}{\dd{t}}}_{t=0}
    \left(\pdv{\Gamma(s,t)}{s}\right),\dot\gamma(s)\right> \dd{s}.
\end{split}
\end{equation}

\begin{lemma}
    \[
        \frac{\mathrm{D}}{\dd{t}}\left( \pdv{\Gamma(s,t)}{s} \right) 
        =\frac{\mathrm{D}}{\dd{s}}\left(\pdv{\Gamma(s,t)}{t}\right)
    .\] 
\end{lemma}
\begin{proof}
Let \((x^1,\ldots,x^n)\) be local coordinate \st\ \(\Gamma(s,t)\) is contained
in this coordinate chart.
\begin{flalign*}
    \implies &\Gamma(s,t)=(x^1(s,t),\ldots,x^n(s,t))
    \implies \begin{cases}
        \pdv{\Gamma}{s}=x^i_s\pdv{x^i} \\
        \pdv{\Gamma}{t}=x^i_t\pdv{x^i}
    \end{cases}. &
\end{flalign*}
\begin{flalign*}
    \implies \frac{\mathrm{D}}{\dd{t}}\left(\pdv{\Gamma}{s}\right)
    &=\frac{\mathrm{D}}{\dd{t}}\left(x^i_s\pdv{x^i}\right) & \\
    &=x^i_{st}\pdv{x^i}+x^i_s x^j_t\nabla_{\pdv{x^i}}\pdv{x^j} \\
    &=\frac{\mathrm{D}}{\dd{s}}\left(\pdv{\Gamma}{t}\right)
.\end{flalign*}
\end{proof}

Then
\begin{flalign*}
    \text{\cref{eq:variation-geodesic}}
    &= \int_{0}^{l}\left<\frac{\mathrm{D}}{\dd{s}}\left(\eval{\pdv{\Gamma}{t}}_{t=0}
    \right),\dot\gamma(s)\right> \dd{s} & \\
    &= \eval{\left<\eval{\pdv{\Gamma}{t}}_{t=0},\dot\gamma(s)\right>}_0^l
    -\int_{0}^{l}\left<\eval{\pdv{\Gamma}{t}}_{t=0},\nabla_{\dot\gamma}
    \dot\gamma\right> \dd{s}
.\end{flalign*}
Note the variational family fix endpoints.
\begin{flalign*}
    \implies & \eval{\pdv{\Gamma}{t}}_{t=0}=\eval{\pdv{\Gamma}{t}}_{t=0}(l)=0 &
\end{flalign*}
Hence
\begin{flalign*}
    \text{\cref{eq:variation-geodesic}}
    &=-\int_{0}^{l}\left<\eval{\pdv{\Gamma}{t}}_{t=0},\nabla_{\dot\gamma}\dot\gamma
    \right> \dd{s} &
\end{flalign*}
Hence Euler-Lagrange equation is \(\nabla_{\dot\gamma}\dot\gamma=0\) (geodesic
equation).

We conclude that if \(\gamma(s)\) is a curve having shortest length between two points
on \(M\), then it must be a geodesic. Conversely, we only have geodesics are local
minimizing curves (prove later).

\begin{example}
    Every great circle on \(\mathbb{S}^2\) is a geodesic.
\end{example}
Note \[
    \nabla_{\dot\gamma}\dot\gamma=0\iff \ddot x^k(t)+\Gamma_{ij}^k(x(t))\dot x^i(t)
    \dot x^j(t)=0.
\] By ODE theory, at each point \(p=(x^1(t_0),\ldots,x^n(t_0))\) with any prescribed
direction \(v=\dot x^k(t_0)\pdv{x^k}\), there is a unique geodesic starting at \(p\)
with initial velocity \(v\). Locally, \ie\ \(\exists\,\gamma(t)\colon (0,\eps)\to M\)
geodesic \st\ \(\gamma(0)=p,\dot\gamma(0)=v\).

\begin{remark}
\begin{enumerate}[(1)]
\item Previously, we have considered the length functional of a family of curves
    \(\Gamma(s,t)=\gamma_t(s)\colon [0,l]\to M\): \[
        L(\gamma_t(s))=\int_{0}^{l}\left|\dot\gamma_t(s)\right|\dd{s}
    .\] In fact, when we consider the Euler-Lagrange equation, it's more convenient
    to consider the ``energy'' functional 
    \begin{equation}\label{eq:energy-functional}
        E(\gamma_t(s))=\frac{1}{2}\int_{0}^{l}\left|\dot\gamma_t(s)\right|^2\dd{s}
    .\end{equation}
    (Then there won't appear denominator term).
\item A generalization of such type of functional: let \(u\colon(M,g)\to (N,h)\)
    be \(C^\infty\) map, the energy functional is \[
        E(u)=\int_{M}\left|\dd{u}\right|^2\dd{V_g}
    .\] If \((x^1,\ldots,x^m)\) is local coordinate of \(M\), \(u(x)=(u^1,\ldots,u^n)
    \), then \[
        \left|\dd{f}\right|_{g,h}^2=g^{ij}h_{\alpha\beta}
        \pdv{u^\alpha}{x^i}\pdv{u^\beta}{x^j}
    .\] As special cases:
    \begin{enumerate}[(a)]
    \item If \(M\) is an interval \([a,b]\), \(u\colon [a,b]\to (N,h)\), \(E(u)\)
        is just~\cref{eq:energy-functional}.
    \item If \(M=\Omega\subset \mathbb{R}^n,N=\mathbb{R}\), \ie\ \(u\) is smooth
        function \(\Omega\to \mathbb{R}\) with compact support, \[
            E(u)=\int_{\mathbb{R}^n}|\dd{f}|^2\dd{x^1}\cdots \dd{x^n}
        .\] Then Euler-Lagrange equation is just harmonic function equation.
    \end{enumerate}
\end{enumerate}
\end{remark}

\begin{definition}
    The solution of Euler-Lagrange equation of~\cref{eq:energy-functional} is called
    harmonic maps.
\end{definition}

\subsection{Complete manifold}
From geodesic equation \(\nabla_{\dot\gamma}\dot\gamma=0\), \(\forall\,p\in M,v\in
T_p M\), \(\exists\) unique geodesic \(\gamma(t)\colon (-\eps,\eps)\to M\) \st\ 
\(\gamma(0)=p,\dot\gamma(0)=v\). In fact we can write \(\gamma(t)=\exp_p(tv)\) locally.

A natural question to ask in how large the defining domain of \(\gamma(t)\) could be.

\noindent{\large\color{blue}\boxed{Observation}} If \(I_1,I_2\) are two interval in
\(\mathbb{R}, t_0\in I_1\cap I_2\), \(\gamma_1\colon I_1\to M,\gamma_2\colon I_2\to M\)
are two geodesics \st\ \(\gamma_1(t_0)=\gamma_2(t_0)\) and \(\dot\gamma_1(t_0)
=\dot\gamma_2(t_0)\). Then \(\gamma_1\) and \(\gamma_2\) agree on \(I_1\cap I_2\) by
uniqueness of solution of ODE\@.

Then \(\gamma_1\cup \gamma_2\) is an extension of geodesics \(\gamma_1\) and
\(\gamma_2\). Furthermore, it is also a geodesic defined on \(I_1\cup I_2\).

\begin{definition}[Maximal geodesic]\hfill\\
    \(\gamma(t)\colon I\to M\) is called a maximal geodesic if \(I\) is the largest
    defining domain of \(\gamma(t)\), \ie\ it cannot be extended to a geodesic on
    a larger interval containing \(I\).
\end{definition}
\begin{remark}
    A maximal geodesic can be defined on
    \begin{enumerate}[(1)]
    \item \(\mathbb{R}\);
    \item Finite open interval \((a,b)\);
    \item Half interval \((-\infty,b)\) or \((a,+\infty)\).
    \end{enumerate}
\end{remark}
\begin{definition}[Complete manifold]\hfill\\
    A Riemannian manifold \((M,g)\) is called (geodesic) complete if all maximal
    geodesics are define on \(\mathbb{R}\).
\end{definition}
\begin{remark}
    This means each geodesic starting from a point \(p\) with initial velocity \(v\)
    can be extended from both sides for all time. Furthermore, \(\exp_p\colon
    T_p M\to M\) is defined for all \(v\in T_p M\).
\end{remark}

\begin{example}
    \(\mathbb{R}^2\setminus\{0\}\) is not geodesic complete. Consider geodesic
    \(\gamma(t)=(t,0)\), which can only be defined on \((-\infty,0)\).
\end{example}

For the completeness of a manifold, we mention the Hopf-Rinow theorem, the proof of
which will be left to Riemannian geometry course.

Note the Riemannian metric \(g\) on \(M\) defines a distance function \[
    d(p,q)=\inf\{L(\gamma):\gamma\text{ piecewise smooth from } p \text{ to }q\}
.\] Then \(M\) became a metric space.

Recall a metric space \((M,d)\) is complete if every Cauchy sequence is convergent.

\begin{theorem}[Hopf-Rinow]\label{thm:hopf-rinow}
    Let \((M,g)\) be a manifold without boundary, TFAE\@:
    \begin{enumerate}[(1)]
    \item \((M,g)\) is geodesic complete.
    \item \((M,g)\) is geodesic complete at some \(p\in M\), \ie\ \(\exp_p\) is defined
        on whole \(T_p M\).
    \item \((M,d)\) is complete metric space.
    \item \(A\subset M\) is compact iff \(A\) is bounded and closed.
    \item Every closed geodesic ball \(\overline{B}(p,r)\) is compact.
    \end{enumerate}
\end{theorem}

\begin{corollary}
    If \((M,g)\) is complete, then any \(p,q\in M\) can be jointed by a minimal
    geodesic, \ie\ the distance \(d(p,q)\) is realized by length of a geodesic.
\end{corollary}
\begin{corollary}
    If \((M,g)\) is closed manifold (compact without boundary), then \((M,g)\) is
    complete. In particular, any surface without boundary in \(\mathbb{R}^3\) which is
    a closed subset is complete.
\end{corollary}

\begin{remark}
    One of the famous incomplete Riemannian metric lies in the study of geometry of
    the moduli space (or Teichm\"uller space) of Riemann surfaces: \[
        T_g=\frac{\{\text{Conformal structure of }\Sigma_g\}}{\text{Orientation
        preserving diffeomorphisms}}
    .\] On \(T_g\), each point represents a Riemann surface of genus  \(g\) with a
    given conformal structure. There is a metric on such space defined by \(L^2\)
    pairing of infinitesimal conformal structure at each point. Such metric is called
    the Weil-Petersson metric an it's not complete.
\end{remark}

\noindent{\color{red}\underline{Key fact:}} Studying the metric completion of
Weil-Petersson metric is related to the compactification of the moduli space!
This is one of many correspondence of Differential Geometry \(\longleftrightarrow\)
Algebraic Geometry.

\subsection{Exponential map \& geodesic spherical coordinate}
Let's first see a rescaling argument of geodesic:

We have seen \(\forall\,p\in M\), along any prescribed velocity \(v\in T_p M\), there
is a unique geodesic \(\gamma(t)\colon [0,\eps)\to M\) \st\ \(\gamma(0)=p,
\dot\gamma(0)=v\). By rescaling the velocity \(v\rightsquigarrow\lambda v\), we obtain
the geodesic \(\beta(\tau)\colon[0,\frac{\eps}{\lambda})\to M\), which has the same
trace with \(\gamma(t)\) and \(\beta(0)=p,\dot\beta(0)=\lambda v\). In particular,
we introduce notation \[
    \gamma(t,v)=\gamma(t),\quad \beta(\tau,\lambda v)=\beta(\tau),\quad \lambda\tau=t
.\] Then \(\beta(\tau,\lambda v)=\beta(\tau)=\gamma(t)=\gamma(t,v)=\gamma(\lambda\tau,v
),\forall\,\tau\in [o,\frac{\eps}{\lambda})\).

This rescaling argument allows us to consider geodesics defined on \([0,1]\).
\begin{definition}[Exponential map]
    Let \(p\in M,v\in T_p M,\gamma\colon[0,1]\to M\) is a geodesic \st\ \(\gamma(0)=p,
    \gamma'(0)=v\). The exponential map is defined as
    \begin{align*}
        \exp_p\colon T_p M &\longrightarrow M \\
        v &\longmapsto \exp_p(v)=\gamma(1)
    .\end{align*}
\end{definition}
Note the length of \(\gamma\) between \(p\) and \(\gamma(1)\) is \(|v|\), and
\(\gamma(t)=\exp_p(tv)\) for \(t\in [0,1]\), \(\exp_p(0)=p\). Hence geodesic \(\gamma
(t)\) is the image of ray from \(0\) in \(T_p M\) under \(\exp_p\).

We state the following fact without proof:
\begin{prop}
    \(\exists\,\eps>0\) \st\ \(\exp_p\colon B(0,\eps)\to M\) is a diffeomorphism
    where \(B(0,\eps)\) is ball of radius \(\eps\) centered at origin in \(T_p M\).
\end{prop}
\begin{remark}
    The radius \(\eps\) could be very large or even \(\infty\), but also could be
    a finite number.
\end{remark}

\begin{example}
    \(\exp_p\colon T_p M\to M\) where \(p\) is north pole in \(\mathbb{S}^2\).
    \(\exp_p(v)=q\) the south pole if \(|v|=\pi\), then \(\exp_p\) fails to be a
    diffeomorphism on \(B(0,r),r>\pi\).
\end{example}

\begin{definition}
    \(W\subset M\) is called a normal neighbourhood of \(p\), if for some \(U\subset 
    T_p M\) containing 0, \(\exp_p\colon U\to W\) is diffeomorphism.
\end{definition}

For normal neighbourhood, we can endow coordinate at \(p\) by considering the image
of Cartesian coordinate and polar coordinate at origin of \(T_p M\cong \mathbb{R}^n\).

\subsubsection*{Normal coordinate}
Let \(e_1,\ldots,e_n\) be orthonormal basis of \(T_p M\cong \mathbb{R}^n\), for any
vector \(v\in U\subset T_p M\), write \(v=x^i e_i\).
Then we say \(q=\exp_p(v)\) has coordinate \((x^1,\ldots,x^n)\). In particular,
\(\exp_p(tv)\) has coordinate \((tx^1,\ldots,tx^n)\) and \(p\) has coordinate \(0\).

Since \(\gamma(t)=\exp_p(tv)\) satisfies geodesic equation, it writes: \[
    \gamma^k(t)=tx^k,\ \Gamma_{ij}^k(\gamma(t))x^i x^j=0\quad
    (\ddot \gamma^k(t)+\Gamma_{ij}^k(\gamma(t))\dot x^i(t)\dot x^j(t)=0)
.\] At \(p=\gamma(0)\), \(\Gamma_{ij}^k(\gamma(0))x^i x^j=0\). Since \(v=x^i e_i\) is
chosen arbitrarily, we conclude that \(\Gamma_{ij}^k=0\) at \(p\).
Moreover, \(g_{ij}(p)=g(e_i,e_j)=\delta_{ij}\).

We summarize that the normal coordinate \((x^1,\ldots,x^n)\) in a neighbourhood of 
\(p\) has the following properties:
\begin{enumerate}[(1)]
\item \(p=(0,\ldots,0)\) ;
\item \(g_{ij}(p)=\delta_{ij}\) ;
\item \(\Gamma_{ij}^k(p)=0\);
\item \(\tensor{R}{_{ij}^k_l}(p)=(\pdv{x^i}\Gamma_{jl}^k-\pdv{x^j}\Gamma_{il}^k)(p)\);
\item \(\triangle_g f(p)=g^{ij}(\partial_i \partial_j f-\Gamma_{ij}^k \partial_k f)(p)
    =\triangle f(p)\).
\end{enumerate}

\subsubsection*{Geodesic spherical coordinate}
Take \((\rho,\theta^1,\ldots,\theta^{n-1})\) as polar coordinate in \(T_p M\cong 
\mathbb{R}^n\). Let \(U\) be normal neighbourhood of \(p\in M\). Then \\
Geodesic sphere \(=\) image of \(\{r=r_0\}\) under \(\exp_p\); \\
Radial geodesic \(=\) image of \(\{\theta=\theta_0\}\) under \(\exp_p\).

\((\rho,\theta^1,\ldots,\theta^{n-1})\) is called the geodesic spherical coordinate
and the vector fields are \(\pdv{\rho},\pdv{\theta^1},\ldots,\pdv{\theta^{n-1}}\).

Note \(\theta=\theta_0\) is a ``straight ray'' and \(\rho\) is just the arc-length
parameter of radial geodesic. Hence in \((\rho,\theta^1,\theta^{n-1})\) coordinate
the coefficient of the Riemannian metric. \[
    g_{11}=g(\pdv{\rho},\pdv{\rho})=1\iff \pdv{\rho}\text{ is a radial unit vector
    filed}
.\] Next we study \(g_{\rho\theta^i}\), we'll show it's 0. Let's argue for surface
since \(\theta=\theta_0\) is geodesic, 
\begin{flalign*}
    \implies & \nabla_{\pdv{\rho}}\pdv{\rho}=0
    \implies \Gamma_{11}^1\pdv{\rho}+\Gamma_{11}^2\pdv{\theta}=0. & \\
    \implies & \Gamma_{11}^1=0, \Gamma_{11}^2=0 \\
    & 0=\Gamma_{11}^1=g^{12}\partial_1 g_{12}=-\frac{1}{\det g}g_{12}\partial_1 g_{12}
.\end{flalign*}
Since \(g^{22}=\frac{1}{\det g}g^{11}\neq 0\), \(\partial_1 g_{12}=0\), \ie\ \(g_{12}\)
is independent of \(\rho\).

Note that \((x_1,x_2)=(\rho\cos\theta,\rho\sin\theta)\), \[
    g_{12}=g(\pdv{\rho},\pdv{\theta})=g(\cos\theta\pdv{x^1}+\sin\theta\pdv{x^2},
    -\rho\sin\theta\pdv{x^1}+\rho\cos\theta\pdv{x^2})
.\] Hence \(g_{12}(\rho,\theta)=\lim_{\rho \to 0} g_{12}(\rho,\theta)=0\).

\begin{exercise}
    Proof \(g_{\rho\theta^i}=0\) for general case.
\end{exercise}

We summarize: In geodesic polar coordinate \((\rho,\theta^1,\ldots,\theta^{n-1})\),
\begin{enumerate}[(1)]
\item \(\pdv{\rho}\) is a unit vector field generating a radical geodesic;
\item \(\pdv{\rho}\perp \pdv{\theta}\), \ie\ radial geodesic is orthogonal to
    geodesic spheres.
\end{enumerate}
These two results are called the \emph{Gauss Lemma}.

Hence in geodesic polar coordinate, the metric is \[
    \dd{s}^2=\dd{\rho}^2+\tilde{g}_{ij}(\rho,\theta)\dd{\theta^i}\dd{\theta^j}
.\] 

\begin{exercise}
    If \(S\) is a surface, \((\rho,\theta)\) is geodesic polar coordinate, show that \[
        \lim_{\rho \to 0} \sqrt{G}=0,\quad
        \lim_{\rho \to 0} (\sqrt{G})_\rho=1
    .\] 
\end{exercise}
\begin{remark}
    In surface case, the Gaussian curvature has expression \[
        K=-\frac{(\sqrt{G})_{\rho\rho}}{\sqrt{G}}
        =-\pdv[2]{\rho}(\log \sqrt{G})
        -(\pdv{\rho}\log\sqrt{G})^2
    .\] 
\end{remark}

{\color{red}
So far, we have seen two useful local expression of Gaussian curvature:
\begin{enumerate}[(1)]
\item \(\dd{s}^2=\dd{r}^2+G(r,\theta)\dd{\theta}^2\implies 
    K=-\frac{(\sqrt{G})_{rr}}{\sqrt{G}}\);
\item \(\dd{s}^2=\lambda(u,v)(\dd{u}^2+\dd{v}^2)
    \implies K=-\frac{1}{2\lambda}\triangle \log\lambda\).
\end{enumerate}
}

\noindent\underline{\bfseries Recall:}
\begin{itemize}
\item \(\mathbb{R}^2\): \(\dd{s^2}=\dd{r^2}+r^2\dd{\theta}^2\implies K=0\).
\item \(\mathbb{S}^2\): \(\dd{s^2}=\dd{r^2}+\sin^2 r\dd{\theta}^2\implies K=1\).
\item \(\mathbb{H}^2\): \(\dd{s^2}=\dd{r^2}+\sinh^2 r\dd{\theta}^2\implies K=-1\).
\end{itemize}
Conversely, if we prescribe the Gaussian curvature \(K=0,\pm1\), then solving
ODE \[
    \begin{cases}
        (\sqrt{G})_{rr}+K\sqrt{G}=0, \\
        \lim_{r\to 0}\sqrt{G}=0,\ \lim_{r\to 0}(\sqrt{G})_r=1
    \end{cases}
.\] Hence, \(\dd{s}^2\) can be written in standard form above. This implies if two
regular surfaces have the same constant Gaussian curvature, they must be locally
isometric. This is \emph{Minding's theorem}.

\subsection{Applications of Geodesic Polar Coordinate}

Recall previously, by comparing the area \(A\) of a small neighbourhood of \(p\in S\)
and the Area \(\overline{A}\) of its Gaussian image, then \[
    |K(p)|=\lim_{A \to 0} \frac{\overline{A}}{A}
.\] We also mentioned two more results:
\begin{enumerate}[I.]
\item Let \(C(r)\) be the image of circle of radius \(r\) centered at 0 in \(T_p S\)
    under exponential map, \ie\ the geodesic circle of radius \(r\), then \[
        K(p)=\lim_{r \to 0} 3\cdot\frac{2\pi r-L(C(r))}{\pi r^3}
    .\]
\item Consider similarly the geodesic disk \(D(r)\), then \[
    K(p)=\lim_{r \to 0} 12\cdot \frac{\pi r^2-A(D(r))}{\pi r^4}
.\]
\end{enumerate}
The proof of I and II both rely on the Taylor's expansion of metric tensor.
\begin{proof}
I. \[
    L(C(r))=\lim_{\eps \to 0} \int_{\eps}^{2\pi-\eps}\sqrt{G}\dd{\theta}
\] Since we care abort \(r\to 0\), we use Taylor's expansion of \(\sqrt{G}\).

Recall that \(\eval{\sqrt{G}}_{r=0}=0,\eval{(\sqrt{G})_r}_{r=0}=1\) and \(K=
-\frac{(\sqrt{G})_{rr}}{\sqrt{G}}\). Hence \[
    (\sqrt{G})_{rr}+K\sqrt{G}=0
,\] and \(\eval{(\sqrt{G})_{rr}}_{r=0}=0\). Take one more derivative we have \[
    (\sqrt{G})_{rrr}+K_r\sqrt{G}+K(\sqrt{G})_{r}=0
.\] Let \(r=0\), we see \[
    K(p)=-\eval{(\sqrt{G})_{rrr}}_{r=0}
.\] Hence the expansion is 
\begin{equation}\label{eqn:geodesic-circle}
    \sqrt{G}(r,\theta)=r-\frac{r^3}{3!}K(p)+o(r^3)
.\end{equation}
Then \(L(C(r))=2\pi r-\frac{r^3}{3}\pi K(p)+o(r^3)\), this gives I.

II.\@Follows by integrating \cref{eqn:geodesic-circle} from \(r=0\) to \(R\). We get \[
    A(R)=\int_{0}^{R}L(C(r))\dd{r}=\pi R^2-\frac{R^4}{12}\pi K(p)+o(R^4)
.\] This gives II.
\end{proof}

\subsection{Geodesics are Locally Length Minimizing}
\(\forall\,p\in S\), take a normal neighbourhood \(U\) of \(p\). The metric in geodesic
polar coordinate is \[
    \dd{s}^2=\dd{r}^2+G(r,\theta)\dd{\theta}^2
.\] By shrinking \(U\), we further require \(U=\{\exp_p(v):v\in B(0,\eps)\subset
T_p S\}\). For any \(q=(r_0,\theta_0)\in U\), we have \(r_0<\eps\), consider all
piecewise smooth curve from \(p\) to \(q\) on \(S\). Let \(\gamma(t)\colon [0,t_0]
\to S\) be such a curve. First observe that if \(\gamma(t)\) is entirely contained
in \(U\), we can let \(\gamma(t)=(r(t),\theta(t))\), then \[
    \gamma'(t)=r'(t)\pdv{r}+\theta'(t)\pdv{\theta}
.\] This gives 
\begin{align*}
    L(\gamma(t))&=\int_{0}^{t_0}\left(|r'(t)|^2+|\theta'(t)|^2 G(r,\theta)\right)
    ^{\frac{1}{2}}\dd{t} \\
    &\ge \int_{0}^{t_0}|r'(t)|\dd{t}\ge \int_{0}^{t_0}r'(t)\dd{d}=r(t_0)=r_0
.\end{align*}
Equality holds iff \(\theta'(t)\equiv 0\). \ie\ \(\gamma\) is the radical ray, which
is the geodesic from \(p\) to \(q\), given by \[
    \gamma(t)=\exp_p(r(t)\pdv{r})
.\] If the curve is not entirely in \(U\), let \(t_1\) be the first time \(\gamma\)
meets the boundary. Then for any \(t_2<t_1\) we know the length of \(\eval{\gamma}_{
[0,t_2]}\) must be greater than the geodesic one, \ie\ \[
    L(\gamma|_{[0,t_2]})=\int_{0}^{t_2}|\gamma'(t)|\dd{t}\ge r(t_2)
.\] Let \(t_2\to t_1\) we see \(L(\gamma)\ge r(t_1)=\eps>r_0\).

We conclude that in \(U\), any piecewise smooth curve has greater length than
(radical) geodesic.

\begin{remark}
\begin{enumerate}[(1)]
\item By the Cauchy problem's solution to geodesic equation \[
    \begin{cases}
        \ddot{x}^k(t)+\Gamma_{ij}^k(t)\dot{x}^i(t)\dot{x}^j(t)=0 \\
        \gamma(0)=p,\ \dot\gamma(0)=v_0
    \end{cases}
.\] There is a unique geodesic starting at \(p\), with initial velocity \(v_0\). Hence
in this arguement, by shrinking \(U\) if necessary, we can see in \(U\), there is
unique minimizing geodesic from \(p\) to \(q\).
\item {\color{blue}Try to think about how large the normal neighbourhood could be.}

    This will be answered after we study the second variation of arc-length 
    and this problem is related to the global geometry of the surface.
\item It's important to study the behavior of a family of geodesics from a point.
    The local picture should be kept in mind is:

    {\color{orange}``The smaller curvature is, the faster geodesics will separate''}
\end{enumerate}
\end{remark}

