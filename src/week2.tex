\section{Local theory of a regular space curve}

\begin{goal}
    Describe a space curve by using geometric quantities.
\end{goal}

\begin{ques}
    How to make a space curve? 
\end{ques}
Starting with a straight line, we can bend it and twist it in a given way to
produce a space curve.
\begin{itemize}
    \item Bending the line \(\longrightarrow\) ``curvature''.
    \item Twisting \(\longrightarrow\) ``torsion''.
    \item Their relations are contained in Frenet formula.
    \item Conversely, fundamental theorem of the local theory of curves
        tells, once we're given two function, \(\kappa(s),\tau(s)\), we can
        describe a unique curve in \(\mathbb{R}^3\) up to a rigid motion,
        \st\ \(\kappa(s)\) is its curvature and \(\tau(s)\) is its torsion.
\end{itemize}

\noindent\underline{\bf Recall:} In Calculus, if \(y=f(x)\) represents a curve, then
\(f''(x)\) tells the convexity of the curve. It measures how fast the velocity
changes. It's also related to how straight line is bent.


Let \(\alpha\colon I\to \mathbb{R}^2\) be a regular plane curve, parametrized by
arc length, \ie\ \(|\alpha'(s)|=1\). Then \(\left<\alpha'(s),\alpha''(s)\right> =0\),
and hence \(\alpha''(s)\perp\alpha'(s)\). For a plane curve, we take normal of the
curve to be counterclockwise \(90^\circ\) rotation of the tangent vector.

% Figure fig:w2-plane-curve-eg here

Let \(N\) be the unit normal vector along \(\alpha(s)\), we have \[
    \left<\alpha''(s),N(s)\right> =\pm|\alpha''(s)|
.\] 
\begin{defn}
    The curvature of a plane curve \(\alpha(s)\) is defined as \[
        \kappa(s)=\left<\alpha''(s),N(s)\right>
    .\] 
\end{defn}
\begin{defn}
    Further we denote \(T\) be the unit tangent vector, then the Frenet equation
    of \(\alpha(s)\) is \[
        \begin{cases}
            T'=\kappa N \\
            N'=-\kappa T
        \end{cases}
    .\] Note that \[
        \left<T',N\right> =\kappa\implies \left<T,N'\right> =-\kappa
    .\] 
\end{defn}

\begin{itemize}
    \item \(\kappa>0\implies \) the point on the curve moves counterclockwise
        direction or say ``to its left''.
    \item \(\kappa<0\implies \) the point on the curve moves clockwise direction
        or say ``to its right''.
\end{itemize}

\begin{ques}
    For the curve in \cref{fig:w2-plane-curve-eg}, can you tell where \(\kappa>0\)
    and where \(\kappa<0\) without doing calculation?
\end{ques}

\begin{remark}
    The sign of the curvature of the plane curve is caused by the direction
    convention of the unit normal vector. This could change according to the
    orientation of a curve.
\end{remark}

Next, we take a look at the geometric meaning of \(|\alpha''(s)|\) at some point
\(\alpha(s_0)\). By definition: \[
    |\alpha''(s_0)|=\lim_{h \to 0} \left|\frac{\alpha'(s_0+h)-\alpha'(s_0)}{h}\right|
.\] 
% May be figure here
We have
\begin{align*}
    |\alpha'(s_0+h)-\alpha'(s_0)|
    &= \left(|\alpha'(s_0+h)|^2+|\alpha'(s_0)|^2-2\left<\alpha'(s_0+h),
    \alpha'(s_0)\right> \right)^{\frac{1}{2}} \\
    &= (2-2\cos\theta_h)^{\frac{1}{2}} \\
    &= (2-2(1-\frac{1}{2}\theta_h^2)+\tilde{o}(\theta_h)^4)^{\frac{1}{2}} \\
    &= (\theta_h^2+\tilde{o}(\theta_h)^4)^{\frac{1}{2}}
.\end{align*}
Hence \[
    |\alpha''(s_0)|=\lim_{h \to 0} \left|\frac{\alpha'(s_0+h)-\alpha'(s_0)}{h}\right|
    =\lim_{h \to 0} \left|\frac{\theta_h}{h}\right|=|\theta'(s_0)|
.\] \ie\ \(|\alpha''(s)|\) measures the changing rate of angle of tangents.
