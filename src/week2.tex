\section{Local theory of a regular space curve}

\begin{goal}
    Describe a space curve by using geometric quantities.
\end{goal}

\begin{question}
    How to make a space curve? 
\end{question}
Starting with a straight line, we can bend it and twist it in a given way to
produce a space curve.
\begin{itemize}
    \item Bending the line \(\longrightarrow\) ``curvature''.
    \item Twisting \(\longrightarrow\) ``torsion''.
    \item Their relations are contained in Frenet formula.
    \item Conversely, fundamental theorem of the local theory of curves
        tells, once we're given two function, \(\kappa(s),\tau(s)\), we can
        describe a unique curve in \(\mathbb{R}^3\) up to a rigid motion,
        \st\ \(\kappa(s)\) is its curvature and \(\tau(s)\) is its torsion.
\end{itemize}

\noindent\underline{\bf Recall:} In Calculus, if \(y=f(x)\) represents a curve, then
\(f''(x)\) tells the convexity of the curve. It measures how fast the velocity
changes. It's also related to how straight line is bent.


Let \(\alpha\colon I\to \mathbb{R}^2\) be a regular plane curve, parametrized by
arc length, \ie\ \(|\alpha'(s)|=1\). Then \(\left<\alpha'(s),\alpha''(s)\right> =0\),
and hence \(\alpha''(s)\perp\alpha'(s)\). For a plane curve, we take normal of the
curve to be counterclockwise \(90^\circ\) rotation of the tangent vector.

\begin{figure}[htp]
    \centering
    \begin{tikzpicture}[x=0.75pt,y=0.75pt,xscale=0.5,yscale=-0.5]
    %Curve Lines [id:da7702103302711949] 
    \draw    (141.28,156.07) .. controls (170.72,-26.63) and (441.91,2.34) .. (448.22,122.65) .. controls (454.52,242.96) and (305.26,214) .. (286.34,287.52) .. controls (267.42,361.04) and (334.69,383.32) .. (399.86,378.87) .. controls (465.03,374.41) and (523.9,347.67) .. (542.82,263.01) ;
    %Straight Lines [id:da3645988582193562] 
    \draw    (140.74,93.57) -- (276.21,9.96) ;
    \draw [shift={(277.91,8.91)}, rotate = 148.32] [color={rgb, 255:red, 0; green, 0; blue, 0 }  ][line width=0.75]    (10.93,-3.29) .. controls (6.95,-1.4) and (3.31,-0.3) .. (0,0) .. controls (3.31,0.3) and (6.95,1.4) .. (10.93,3.29)   ;
    %Straight Lines [id:da28401015192032575] 
    \draw    (209.33,51.24) -- (166.04,-19.96) ;
    \draw [shift={(165,-21.67)}, rotate = 58.7] [color={rgb, 255:red, 0; green, 0; blue, 0 }  ][line width=0.75]    (10.93,-3.29) .. controls (6.95,-1.4) and (3.31,-0.3) .. (0,0) .. controls (3.31,0.3) and (6.95,1.4) .. (10.93,3.29)   ;
    %Straight Lines [id:da8270103736877523] 
    \draw    (283.3,308.56) -- (361.67,311.59) ;
    \draw [shift={(363.67,311.67)}, rotate = 182.21] [color={rgb, 255:red, 0; green, 0; blue, 0 }  ][line width=0.75]    (10.93,-3.29) .. controls (6.95,-1.4) and (3.31,-0.3) .. (0,0) .. controls (3.31,0.3) and (6.95,1.4) .. (10.93,3.29)   ;
    %Straight Lines [id:da9257450334110999] 
    \draw    (288.56,235.59) -- (278.19,379.53) ;
    \draw [shift={(278.05,381.53)}, rotate = 274.12] [color={rgb, 255:red, 0; green, 0; blue, 0 }  ][line width=0.75]    (10.93,-3.29) .. controls (6.95,-1.4) and (3.31,-0.3) .. (0,0) .. controls (3.31,0.3) and (6.95,1.4) .. (10.93,3.29)   ;
    %Straight Lines [id:da2500290048593601] 
    \draw    (513.8,327.07) -- (453.9,279.58) ;
    \draw [shift={(452.33,278.33)}, rotate = 38.41] [color={rgb, 255:red, 0; green, 0; blue, 0 }  ][line width=0.75]    (10.93,-3.29) .. controls (6.95,-1.4) and (3.31,-0.3) .. (0,0) .. controls (3.31,0.3) and (6.95,1.4) .. (10.93,3.29)   ;
    %Straight Lines [id:da16103254249771015] 
    \draw    (474.39,378.31) -- (552,277.41) ;
    \draw [shift={(553.22,275.82)}, rotate = 127.57] [color={rgb, 255:red, 0; green, 0; blue, 0 }  ][line width=0.75]    (10.93,-3.29) .. controls (6.95,-1.4) and (3.31,-0.3) .. (0,0) .. controls (3.31,0.3) and (6.95,1.4) .. (10.93,3.29)   ;
    \end{tikzpicture}
    \caption{
        Example of a plane curve and its tangent and normal
    }\label{fig:w2-plane-curve-eg}
\end{figure}

Let \(N\) be the unit normal vector along \(\alpha(s)\), we have \[
    \left<\alpha''(s),N(s)\right> =\pm|\alpha''(s)|
.\] 
\begin{definition}
    The curvature of a plane curve \(\alpha(s)\) is defined as \[
        \kappa(s)=\left<\alpha''(s),N(s)\right>
    .\] 
\end{definition}
\begin{definition}
    Further we denote \(T\) be the unit tangent vector, then the Frenet equation
    of \(\alpha(s)\) is \[
        \begin{cases}
            T'=\kappa N \\
            N'=-\kappa T
        \end{cases}
    .\] Note that \[
        \left<T',N\right> =\kappa\implies \left<T,N'\right> =-\kappa
    .\] 
\end{definition}

\begin{itemize}
    \item \(\kappa>0\implies \) the point on the curve moves counterclockwise
        direction or say ``to its left''.
    \item \(\kappa<0\implies \) the point on the curve moves clockwise direction
        or say ``to its right''.
\end{itemize}

\begin{question}
    For the curve in \cref{fig:w2-plane-curve-eg}, can you tell where \(\kappa>0\)
    and where \(\kappa<0\) without doing calculation?
\end{question}

\begin{remark}
    The sign of the curvature of the plane curve is caused by the direction
    convention of the unit normal vector. This could change according to the
    orientation of a curve.
\end{remark}

Next, we take a look at the geometric meaning of \(|\alpha''(s)|\) at some point
\(\alpha(s_0)\). By definition: \[
    |\alpha''(s_0)|=\lim_{h \to 0} \left|\frac{\alpha'(s_0+h)-\alpha'(s_0)}{h}\right|
.\] 
% No figure here :(
We have
\begin{align*}
    |\alpha'(s_0+h)-\alpha'(s_0)|
    &= \left(|\alpha'(s_0+h)|^2+|\alpha'(s_0)|^2-2\left<\alpha'(s_0+h),
    \alpha'(s_0)\right> \right)^{\frac{1}{2}} \\
    &= (2-2\cos\theta_h)^{\frac{1}{2}} \\
    &= (2-2(1-\frac{1}{2}\theta_h^2)+\tilde{o}(\theta_h)^4)^{\frac{1}{2}} \\
    &= (\theta_h^2+\tilde{o}(\theta_h)^4)^{\frac{1}{2}}
.\end{align*}
Hence \[
    |\alpha''(s_0)|=\lim_{h \to 0} \left|\frac{\alpha'(s_0+h)-\alpha'(s_0)}{h}\right|
    =\lim_{h \to 0} \left|\frac{\theta_h}{h}\right|=|\theta'(s_0)|
.\] \ie\ \(|\alpha''(s)|\) measures the changing rate of angle of tangents.

In fact, for a plane curve, let \(\theta\) be the angle between \(\alpha'(s_0)\) and
\(\alpha'(s)\), then \[
    \left<\alpha'(s),\alpha'(s_0)\right> =\cos\theta_s\implies 
    \left<\alpha''(s),\alpha'(s_0)\right> =-\sin\theta_s\cdot \theta_s'
.\] Notice that \(\cos\theta_s\) is the projection of \(\alpha'(s_0)\) on the tangent
\(\alpha'(s)\), hence \[
    \sin\theta_s=\left<\alpha'(s_0),N(s)\right> 
.\] On the other hand, \(\alpha''(s)=T'(s)=\kappa(s)N(s)=\pm|\alpha''(s)|N(s)\),
this gives \(\theta_s'=\pm|\alpha''(s)|\).


Let \(\alpha\colon I\to \mathbb{R}^3\) be a space curve, parametrized by arclength,
\ie\ \(|\alpha'(s)=1|\), we also have \(\left<\alpha'(s),\alpha''(s)\right> =0\),
\ie\ \(\alpha''(s)\perp\alpha'(s)\).

Unlike case of dim 2, it does not make sense to prescribe a normal vector of
a curve. However, from above discussion, we see the geometric meaning of
\(|\alpha''(s)|\) is the measure of how fast the point on the curve leaving the
straight line. We came into following definition:
\begin{definition}
    The \emph{curvature} of a regular space curve \(\alpha(s)\) parametrized by
    arclength is defined as \[
        \kappa(s)=|\alpha''(s)|
    .\] And the unit normal vector at \(\alpha(s)\) is \[
        N=\frac{\alpha''(s)}{|\alpha''(s)|},\quad\text{for }|\alpha''(s)|>0
    .\] 
\end{definition}
\begin{remark}\hfill
\begin{itemize}
    \item If \(|\alpha''(s)|\equiv 0\) then \(\alpha\) must be a straight line,
        and all unit normal vectors line on a unit circle \(\perp\alpha\).
    \item If \(|\alpha''(s_0)|=0\), we call \(s_0\) a singular point of order 1.
        (Note. \(s_0\) \st\ \(|\alpha(s_0)|=0\) is called a singular point 
        of order 0) At such points, there is no well-defined normal vector.
    \end{itemize}
\end{remark}

\begin{definition}
The plane determined by \(T,N\) is called the \emph{osculating plane} of
\(\alpha(s)\). The unit normal vector of the osculating plane
\[
    B=T\times N
\] is called \emph{binormal vector}.
\end{definition}
\begin{remark}\hfill
\begin{itemize}
    \item \(\{T,N,B\}\) satisfies the right hand rule.
    \item \(|B'|\) measures how fast the point leaves the osculating plane.
\end{itemize}
If we denote \(\theta_h\) be angle between \(B(s_0+h)\) and \(B(s_0)\), similar to
former calculation, we have
\begin{align*}
    |B'(s_0)|&= \lim_{h \to 0} \left|\frac{B(s_0+h)-B(s_0)}{h}\right| \\
    &= \lim_{h \to 0} \left|\frac{\sqrt{2-2\cos\theta_h}}{h}\right| \\
    &= |\theta_{s_0}'|
.\end{align*}
\end{remark}

As we saw, at each (non-singular) point on a space curve \(\alpha(s)\), we can
associate an oriented orthonormal frame \(\{T,N,B\}\).
\begin{question}
    How these three vector fields are related to the geometry of the curve?
\end{question}
By definition, we write out 0-order info of \(\{T,N,B\}\), \ie\ \[
    \begin{cases}
        T=\alpha' & \\
        N=\frac{\alpha''}{|\alpha''|} & \implies T'=\alpha''=\kappa N \\
        B=T\times N & \implies B'=T'\times N+T\times N'=T\times N'
    \end{cases}
.\] Since \(|B|=1\), we have \(B'\perp B\). Also from above we see \(B'\perp T\),
hence \(B'\parallel N\).
\begin{definition}
    We define \[
        B'=\tau N
    .\] Here \(\tau(s)\) is called the torsion of \(\alpha(s)\).
\end{definition}

Next, we also want to find \(N'\). \(N=B\times T\) gives
\begin{align*}
    N'&= B'\times T+B\times T' \\
    &= \tau N\times T+B\times (\kappa N) \\
    &= -\kappa T-\tau B
.\end{align*}
\begin{theorem}
    The fundamental equations of a space curve (also called the Frenet equations)
    is given by
    \begin{equation}\label{eq:frenet3}
        \begin{bmatrix}
            T'\\ N'\\ B'
        \end{bmatrix}=\begin{bmatrix}
            0 & \kappa & 0 \\
            -\kappa & 0 & -\tau \\
            0 & \tau & 0
        \end{bmatrix}\begin{bmatrix}
            T\\ N\\ B
        \end{bmatrix}
    .\end{equation}
\end{theorem}

\begin{remark}\hfill
\begin{enumerate}[(1)]
    \item \(\tau\equiv 0\) but \(\kappa\neq 0\) at all points \(\implies \alpha(s)\)
    is a plane curve. (Note this may be not true if we don't assume \(\kappa\neq 0\),
    see ex.10 in Do Carmo's book) \\
    \(\kappa\equiv 0\implies \alpha(s)\) is a straight line.

    Note \(\tau\equiv 0\implies B'=0\implies B\) is constant vector. Further, \[
        (\alpha\cdot B)'=\alpha'\cdot B=T\cdot B=0,
    \] gives \(\alpha\cdot B=\text{constant}\implies (\alpha(s)-\alpha(s_0))
    \cdot B=0\). Since \(\kappa\neq 0\) at all points, the osculating plane
    is always well-defined, hence \(B\) is always defined, we proved \(\alpha\) lie
    in some plane perpendicular to vector \(B\).
    \item In different textbooks, you may see the definition of \(\tau\) having a
        different sign from here.
\end{enumerate}
\end{remark}

\itshape{}
Friendly warning: When studying the Geometry, (even later in Riemannian Geometry),
it happens a lot that different authors use different sign convention for the same
definition. It's very important that you should fix your own notation, ant keep it
consistently!
\normalfont{}

\begin{definition}
    \(\{T,N,B\}\) is called Frenet trihedron of \(\alpha(s)\), it gives an moving 
    orthonormal basis of \(\mathbb{R}^3\) along the curve \(\alpha(s)\).
\end{definition}
The Frenet equation describes how such moving orthonormal basis moves along 
\(\alpha(s)\).

\begin{remark}
    Note that in above discussion, we have chosen a special parameter, the arclength
    parameter, of \(\alpha(s)\). In the study of Geometry, finding a good
    parametrization can simplify a lot of work and itself an important problem. In
    more general framework, it's called a ``Gauge related'' problem.
\end{remark}

We have seen that given a regular curve \(\alpha(s)\), parametrized by arclength,
the Frenet equation is \cref{eq:frenet3}, for some functions \(\kappa(s)>0\) be
its curvature and \(\tau(s)\) be its torsion. Conversely, we ask
\begin{question}
    If we're given smooth functions \(\kappa(s),\tau(s)\) with \(\kappa(s)>0\),
    \begin{enumerate}[(1)]
        \item (Existence) Does there exist a regular curve \(\alpha(s)\) \st\ 
            \(\kappa(s)\) is its curvature and \(\tau(s)\) is its torsion?
        \item (Uniqueness) If such curve exists, is it unique in some sense?
    \end{enumerate}
\end{question}
The answer is {\large\bfseries YES!}

\begin{theorem}[Fundamental theorem of the local theory of curves]
    Let \(\kappa(s),\tau(s)\colon I\to \mathbb{R}\) be smooth functions, assume
    \(\kappa(s)>0\), then
    \begin{itemize}
        \item (Existence) There is a regular curve realize \(\kappa\) and \(\tau\)
            as its curvature and torsion.
        \item (Uniqueness) If \(\alpha,\beta\) are two such curves parametrized by
            arclength parameter, then they only differ by a rigid motion of
            \(\mathbb{R}^3\). \ie\ \(\exists\,T\in O(3),c\in \mathbb{R}^3\) \st\ 
            \(\beta=T\alpha+c\).
    \end{itemize}
\end{theorem}
\begin{remark}\hfill
\begin{enumerate}[(1)]
    \item Existence follows from a Cauchy problem (initial value problem) of
        ODE system.
    \item The curve is unique up to a rotation of \(\mathbb{R}^3\) and a translation.
\end{enumerate}
\end{remark}

\begin{proof}
    If we denote \[
        X(s)=\begin{bmatrix}
            T \\ N \\ B
        \end{bmatrix},\ 
        P(s)=\begin{bmatrix}
            & \kappa & \\
            -\kappa & & -\tau \\
            & \tau & 
        \end{bmatrix}
    .\] Where \(T,N,B\) are viewed as \textbf{row vectors}. Then the Frenet equation
    writes as \[
        X'=PX
    .\] This is a first order linear ODE (of nine unknown functions), then by the
    existence and uniqueness theorem of ODEs, given any initial value \[
        X(0)=\begin{bmatrix}
            T_0 \\ N_0 \\ B_0
        \end{bmatrix}
    \] witch form an orthonormal basis, the system has a unique solution that extend
    to whole domain \(I\).

    We need to check the solution actually is orthonormal frame for each \(s\),
    notice the orthonormal relation can be written as \[
        XX^t=I_3
    .\] Where \(I_3\) is the identity matrix of dimension three. Take differential
    on left hand side of the equation we get
    \begin{align*}
        \frac{\dif}{\dif s}(XX^t)&= X'X^t+X(X^t)'=X'X^t+X(X')^t \\
        &= PXX^t+X(PX)^t \\
        &= PXX^t+XX^t P^t \\
        &= PXX^t-XX^t P
    .\end{align*}
    If we denote \(Y=XX^t\), we see \[
        Y'=PY-YP
    \] gives again a first order ODE system, with initial value \(Y(0)=I_3\). But
    there is an obvious solution \(Y\equiv I_3\), so by uniqueness theorem, this
    is it. This proves \(XX^t=I_3\) for any \(s\).

    Until now, we have proved the existence of orthonormal moving frame
    \(\{T,N,B\}\). Notice \(T\) is just \(\alpha'(s)\), so given initial point
    \(\alpha(0)=\alpha_0\), integrate w.r.t \(s\) gives a valid solution \[
        \alpha(s)=\alpha_0+\int_{0}^{s}T(\xi)\dif \xi
    .\]

    For the uniqueness, we need to look carefully into the initial condition we
    chose for the solution \(\alpha\), that is, choice of initial frame
    \(\{T_0,N_0,B_0\}\) and initial point \(\alpha_0\). Given two valid solution
    curve \(\alpha,\beta\), with initial condition \((X_a,\alpha_0)\) and \((X_b,
    \beta_0)\), we choose an orthogonal matrix \(T=X_b X_a^{-1}\), a constant
    \(c=\beta_0-T\alpha_0\), then we see the curve \[
        \tilde{\beta}=T(\alpha-\alpha_0)+\beta_0=T\alpha+c
    \] satisfy exactly the same initial condition as \(\beta\), so they must agree.
    This proves the uniqueness up to rigid motion we stated before.
\end{proof}

\begin{remark}\hfill
\begin{enumerate}[(1)]
    \item \textbf{Exercise:} Check that for solution given above, \(\kappa\) and
        \(\tau\) are its curvature and torsion.
    \item We can view the ODE problem at a somehow higher point. Consider the space
        of all orthonormal frames, it is actually a smooth manifold. It's a little
        non-trivial but we can identity the space with three dimensional rotation
        group \(SO(3)\), smoothly embedded into \(\mathbb{R}^9\), the space of
        three dimensional matrices. The equation, can be interpreted to a (time
        dependent) vector field on \(SO(3)\). One can verify the vector field
        is tangent to the manifold, so it is actually a vector field not only in
        \(\mathbb{R}^9\) but in \(SO(3)\) itself. Similar to the existence and
        uniqueness theorem of ODEs on Euclid spaces, we have a version of such
        theorem for smooth manifolds. It states that for a smooth manifold \(M\),
        a (maybe time dependent) smooth vector field \(X\) on \(M\), then with any
        given initial point \(p\), there exists an integral curve on \(M\) starting
        at \(p\), tangent to \(X\) everywhere, and it is unique. Using the theorem,
        we can say that with given initial \(\{T_0,N_0,B_0\}\), there exists a
        unique solution \(\{T(s),N(s),B(s)\}\). Note that the solution is
        automatically lie on \(SO(n)\), no need to verify it is orthonormal.
\end{enumerate}
\end{remark}
