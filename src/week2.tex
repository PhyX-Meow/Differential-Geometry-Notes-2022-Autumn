\section{Local theory of a regular space curve}

\begin{goal}
    Describe a space curve by using geometric quantities.
\end{goal}

\begin{ques}
    How to make a space curve? 
\end{ques}
Starting with a straight line, we can bend it and twist it in a given way to
produce a space curve.
\begin{itemize}
    \item Bending the line \(\longrightarrow\) ``curvature''.
    \item Twisting \(\longrightarrow\) ``torsion''.
    \item Their relations are contained in Frenet formula.
    \item Conversely, fundamental theorem of the local theory of curves
        tells, once we're given two function, \(\kappa(s),\tau(s)\), we can
        describe a unique curve in \(\mathbb{R}^3\) up to a rigid motion,
        \st\ \(\kappa(s)\) is its curvature and \(\tau(s)\) is its torsion.
\end{itemize}

\noindent\underline{\bf Recall:} In Calculus, if \(y=f(x)\) represents a curve, then
\(f''(x)\) tells the convexity of the curve. It measures how fast the velocity
changes. It's also related to how straight line is bent.


Let \(\alpha\colon I\to \mathbb{R}^2\) be a regular plane curve, parametrized by
arc length, \ie\ \(|\alpha'(s)|=1\). Then \(\left<\alpha'(s),\alpha''(s)\right> =0\),
and hence \(\alpha''(s)\perp\alpha'(s)\). For a plane curve, we take normal of the
curve to be counterclockwise \(90^\circ\) rotation of the tangent vector.

\begin{figure}[htp]
    \centering
    \begin{tikzpicture}[x=0.75pt,y=0.75pt,xscale=0.5,yscale=-0.5]
    %Curve Lines [id:da7702103302711949] 
    \draw    (141.28,156.07) .. controls (170.72,-26.63) and (441.91,2.34) .. (448.22,122.65) .. controls (454.52,242.96) and (305.26,214) .. (286.34,287.52) .. controls (267.42,361.04) and (334.69,383.32) .. (399.86,378.87) .. controls (465.03,374.41) and (523.9,347.67) .. (542.82,263.01) ;
    %Straight Lines [id:da3645988582193562] 
    \draw    (140.74,93.57) -- (276.21,9.96) ;
    \draw [shift={(277.91,8.91)}, rotate = 148.32] [color={rgb, 255:red, 0; green, 0; blue, 0 }  ][line width=0.75]    (10.93,-3.29) .. controls (6.95,-1.4) and (3.31,-0.3) .. (0,0) .. controls (3.31,0.3) and (6.95,1.4) .. (10.93,3.29)   ;
    %Straight Lines [id:da28401015192032575] 
    \draw    (209.33,51.24) -- (166.04,-19.96) ;
    \draw [shift={(165,-21.67)}, rotate = 58.7] [color={rgb, 255:red, 0; green, 0; blue, 0 }  ][line width=0.75]    (10.93,-3.29) .. controls (6.95,-1.4) and (3.31,-0.3) .. (0,0) .. controls (3.31,0.3) and (6.95,1.4) .. (10.93,3.29)   ;
    %Straight Lines [id:da8270103736877523] 
    \draw    (283.3,308.56) -- (361.67,311.59) ;
    \draw [shift={(363.67,311.67)}, rotate = 182.21] [color={rgb, 255:red, 0; green, 0; blue, 0 }  ][line width=0.75]    (10.93,-3.29) .. controls (6.95,-1.4) and (3.31,-0.3) .. (0,0) .. controls (3.31,0.3) and (6.95,1.4) .. (10.93,3.29)   ;
    %Straight Lines [id:da9257450334110999] 
    \draw    (288.56,235.59) -- (278.19,379.53) ;
    \draw [shift={(278.05,381.53)}, rotate = 274.12] [color={rgb, 255:red, 0; green, 0; blue, 0 }  ][line width=0.75]    (10.93,-3.29) .. controls (6.95,-1.4) and (3.31,-0.3) .. (0,0) .. controls (3.31,0.3) and (6.95,1.4) .. (10.93,3.29)   ;
    %Straight Lines [id:da2500290048593601] 
    \draw    (513.8,327.07) -- (453.9,279.58) ;
    \draw [shift={(452.33,278.33)}, rotate = 38.41] [color={rgb, 255:red, 0; green, 0; blue, 0 }  ][line width=0.75]    (10.93,-3.29) .. controls (6.95,-1.4) and (3.31,-0.3) .. (0,0) .. controls (3.31,0.3) and (6.95,1.4) .. (10.93,3.29)   ;
    %Straight Lines [id:da16103254249771015] 
    \draw    (474.39,378.31) -- (552,277.41) ;
    \draw [shift={(553.22,275.82)}, rotate = 127.57] [color={rgb, 255:red, 0; green, 0; blue, 0 }  ][line width=0.75]    (10.93,-3.29) .. controls (6.95,-1.4) and (3.31,-0.3) .. (0,0) .. controls (3.31,0.3) and (6.95,1.4) .. (10.93,3.29)   ;
    \end{tikzpicture}
    \caption{
        Example of a plane curve and its tangent and normal
    }\label{fig:w2-plane-curve-eg}
\end{figure}

Let \(N\) be the unit normal vector along \(\alpha(s)\), we have \[
    \left<\alpha''(s),N(s)\right> =\pm|\alpha''(s)|
.\] 
\begin{defn}
    The curvature of a plane curve \(\alpha(s)\) is defined as \[
        \kappa(s)=\left<\alpha''(s),N(s)\right>
    .\] 
\end{defn}
\begin{defn}
    Further we denote \(T\) be the unit tangent vector, then the Frenet equation
    of \(\alpha(s)\) is \[
        \begin{cases}
            T'=\kappa N \\
            N'=-\kappa T
        \end{cases}
    .\] Note that \[
        \left<T',N\right> =\kappa\implies \left<T,N'\right> =-\kappa
    .\] 
\end{defn}

\begin{itemize}
    \item \(\kappa>0\implies \) the point on the curve moves counterclockwise
        direction or say ``to its left''.
    \item \(\kappa<0\implies \) the point on the curve moves clockwise direction
        or say ``to its right''.
\end{itemize}

\begin{ques}
    For the curve in \cref{fig:w2-plane-curve-eg}, can you tell where \(\kappa>0\)
    and where \(\kappa<0\) without doing calculation?
\end{ques}

\begin{remark}
    The sign of the curvature of the plane curve is caused by the direction
    convention of the unit normal vector. This could change according to the
    orientation of a curve.
\end{remark}

Next, we take a look at the geometric meaning of \(|\alpha''(s)|\) at some point
\(\alpha(s_0)\). By definition: \[
    |\alpha''(s_0)|=\lim_{h \to 0} \left|\frac{\alpha'(s_0+h)-\alpha'(s_0)}{h}\right|
.\] 
% No figure here :(
We have
\begin{align*}
    |\alpha'(s_0+h)-\alpha'(s_0)|
    &= \left(|\alpha'(s_0+h)|^2+|\alpha'(s_0)|^2-2\left<\alpha'(s_0+h),
    \alpha'(s_0)\right> \right)^{\frac{1}{2}} \\
    &= (2-2\cos\theta_h)^{\frac{1}{2}} \\
    &= (2-2(1-\frac{1}{2}\theta_h^2)+\tilde{o}(\theta_h)^4)^{\frac{1}{2}} \\
    &= (\theta_h^2+\tilde{o}(\theta_h)^4)^{\frac{1}{2}}
.\end{align*}
Hence \[
    |\alpha''(s_0)|=\lim_{h \to 0} \left|\frac{\alpha'(s_0+h)-\alpha'(s_0)}{h}\right|
    =\lim_{h \to 0} \left|\frac{\theta_h}{h}\right|=|\theta'(s_0)|
.\] \ie\ \(|\alpha''(s)|\) measures the changing rate of angle of tangents.

In fact, for a plane curve, let \(\theta\) be the angle between \(\alpha'(s_0)\) and
\(\alpha'(s)\), then \[
    \left<\alpha'(s),\alpha'(s_0)\right> =\cos\theta_s\implies 
    \left<\alpha''(s),\alpha'(s_0)\right> =-\sin\theta_s\cdot \theta_s'
.\] Notice that \(\cos\theta_s\) is the projection of \(\alpha'(s_0)\) on the tangent
\(\alpha'(s)\), hence \[
    \sin\theta_s=\left<\alpha'(s_0),N(s)\right> 
.\] On the other hand, \(\alpha''(s)=T'(s)=\kappa(s)N(s)=\pm|\alpha''(s)|N(s)\),
this gives \(\theta_s'=\pm|\alpha''(s)|\).


Let \(\alpha\colon I\to \mathbb{R}^3\) be a space curve, parametrized by arclength,
\ie\ \(|\alpha'(s)=1|\), we also have \(\left<\alpha'(s),\alpha''(s)\right> =0\),
\ie\ \(\alpha''(s)\perp\alpha'(s)\).

Unlike case of dim 2, it does not make sense to prescribe a normal vector of
a curve. However, from above discussion, we see the geometric meaning of
\(|\alpha''(s)|\) is the measure of how fast the point on the curve leaving the
straight line. We came into following definition:
\begin{defn}
    The \emph{curvature} of a regular space curve \(\alpha(s)\) parametrized by
    arclength is defined as \[
        \kappa(s)=|\alpha''(s)|
    .\] And the unit normal vector at \(\alpha(s)\) is \[
        N=\frac{\alpha''(s)}{|\alpha''(s)|},\quad\text{for }|\alpha''(s)|>0
    .\] 
\end{defn}
\begin{remark}\hfill
\begin{itemize}
    \item If \(|\alpha''(s)|\equiv 0\) then \(\alpha\) must be a straight line,
        and all unit normal vectors line on a unite circle \(\perp\alpha\).
    \item If \(|\alpha''(s_0)|=0\), we call \(s_0\) a singular point of order 1.
        (Note. \(s_0\) \st\ \(|\alpha(s_0)|=0\) is called a singular point 
        of order 0) At such points, there is no well-defined normal vector.
    \end{itemize}
\end{remark}

\begin{defn}
The plane determined by \(T,N\) is called the \emph{osculating plane} of
\(\alpha(s)\). The unite normal vector of the osculating plane
\[
    B=T\times N
\] is called \emph{binormal vector}.
\end{defn}
\begin{remark}\hfill
\begin{itemize}
    \item \(\{T,N,B\}\) satisfies the right hand rule.
    \item \(|B'|\) measures how fast the point leaves the osculating plane.
\end{itemize}
\end{remark}
