\begin{definition}
    \[\nabla\colon \Gamma(TM)\times \Gamma (TM)\to \Gamma(TM)\]
    \[(X,Y)\mapsto \nabla_X Y\]
    is called the affine connection on \(M\).
\end{definition}
\underline{Fact}: There are many such affine connections!

In practice, we need local expressions of \(\nabla_V W\). Let 
\((x^1,\ldots,x^n)\) be a local coordinate on \(U\subset M\), 
\(V=V^i\pdv{x^i},W=W^i\pdv{x^i}\), then
\begin{align*}
    \nabla_V W &=\nabla_{V^i\pdv{x^i}}\left(W^j\pdv{x^j}\right)\\
    &= V^i\nabla_{\pdv{x^i}}\left(W^j \pdv{x^j}\right)\\
    &=V^i\left(\left(\nabla_{\pdv{x^i}} W^j\right)\pdv{x^j}
    + W^j \nabla_{\pdv{x^i}} \pdv{x^j}\right)\\
    &=V^i\left(\pdv{W^j}{x^i}\pdv{x^j}+W^j
    \boxed{\nabla_{\pdv{x^i}}\pdv{x^j}}\right).
\end{align*}
\begin{definition}
    On \(U\), we define the Christoffel symbol by 
    \[
        \nabla_{\pdv{x^i}}\pdv{x^j}=\Gamma\indices*{_{ij}^k}\pdv{x^k}.
    \]
    (Here, one should compare this definition with the Christoffel
    symbol introduced in the study of equation of motion 
    in previous lectures).
\end{definition}
\[
    \Rightarrow \nabla_V W= V^i\left(\pdv{W^j}{x^i}+W^k\Gamma
    \indices*{_{ki}^j}\right)\pdv{x^j}   .
\]
Conventionally, we define
\[
    \nabla_i W^j=\pdv{W^j}{x^i}+\Gamma\indices*{_{ik}^j}W^k    
\]
\begin{itemize}
    \item \(\nabla_i W^j\): Taking covariant derivative 
    of \(j\)-th component of \(W\) along the \(i\)-th coordinate
    direction.
    \item \(\pdv{W^j}{x^i}\): Euclidean derivative.
    \item \(\Gamma\indices*{_{ik}^j}W^k\): Correction term.
\end{itemize}
This notion is frequently used in geometry references.

Again, because there are tons of selection of affine connections,
this yields many choices of \(\Gamma\indices*{_{ij}^k}\)!
After we introduce the Riemannian metric, we shall see there is a unique
affine connection compatible with the Riemannian metric.
\begin{definition}[Riemannian metric]
    Let \(M\) be a smooth manifold. A Riemannian metric on \(M\)
    is a smooth map 
    \[
     g\colon \Gamma(TM)\times \Gamma(TM)\to C^\infty(M),
    \]
    such that 
    \(\forall X,Y,Z\in \Gamma(TM), f\in C^\infty(M)\)
    there is the following 
    \begin{enumerate}[(1)]
        \item \(g(X,Y)=g(Y,X)\).(symmetry)
        \item \(g(X,X)\ge 0\), and equality achieves iff \(X=0\).
        (positive definite)
        \item \(g(f X+Y,Z)=f\cdot g(X,Z)+g(Y,Z)\).(\(C^\infty\) linearity) 
    \end{enumerate}
    \((M,g)\) is called a Riemannian manifold.
\end{definition}
\begin{remark}
    At each \(p\in M\), \(g\) defines an inner product \(g_p\)
    on \(T_p M\). If we choose a coordinate chart near \(p\)
    with local coordinate \((x^1,\ldots,x^n)\),
    the local expression of \(g\) is written as 
    \[
        g=g_{ij}dx^i dx^j    ,
    \]
    where \(g_{ij}=g\left(\pdv{x^i},\pdv{x^j}\right)\).
\end{remark}
\begin{example}
    The \engordnumber{1} fundamental form on \(M\) is a Riemannian metric.
\end{example}
Using the Riemannian metric \(g\), we can define 
the length, area, angle, et.c.
\begin{definition}
    \begin{enumerate}[(1)]
        \item \(X\in \Gamma(TM)\), \(|X|=\sqrt{g(X,X)}\).
        \item The volume density on \((M,g)\) is defined as 
        \[
            \dd V=\sqrt{\det(g_{ij})} dx^1\wedge\ldots \wedge dx^n .
        \]
        \item The volume of a bounded region \(B\) is 
        \[
            V(B)=\int_B 1 \dd V
        \]
    \end{enumerate}
\end{definition}
With the Riemannian metric introduced, we can uniquely determine
an affine connection compatible with the Riemannian metric.
\begin{theorem}
    Let \((M,g)\) be a Riemannian manifold, then there is a unique
    affine connection 
    \(\nabla\colon \Gamma(TM)\times \Gamma (TM)\to \Gamma(TM)\)
    satisfying following conditions: \(\forall X,Y,Z\)
    \begin{enumerate}[(1)]
        \item \(\nabla_Z \left(g(X,Y)\right)=g\left(
            \nabla_Z X,Y\right)+g(X,\nabla_Z Y)\).(Compatible with metric 
            \(g\))\footnotemark
        \item \(\nabla_X Y-\nabla_Y X=[X,Y]\).(Torsion free)
    \end{enumerate}
    \footnotetext{(1) is also equivalent to \(\nabla g=0\). 
    (We shall talk about this later)}
    The connection \(\nabla\) is called Levi-Civita connection.
\end{theorem}
Locally, take \(X=\pdv{x^i},Y=\pdv{x^j},Z=\pdv{x^k}\)
\begin{align*}
    (1)&\Rightarrow \pdv{g_{ij}}{x^k}=\Gamma\indices*{_{ki}^l}g_{lj}
    +\Gamma\indices*{_{kj}^l}g_{l i}    \\
    (2)&\Rightarrow \Gamma\indices*{_{ij}^k}=\Gamma\indices*{_{ji}^k}.
\end{align*}
\begin{exercise}
    Show that 
    \[
        \Gamma\indices*{_{ij}^k}=\frac{1}{2}g^{kl}\left(
            \pdv{g_{jl}}{x^i}+\pdv{g_{il}}{x^j}-\pdv{g_{ij}}{x^l}
        \right)    
    \]
    This is just the same expression as we see on.
    %打到这里的时候page142还不存在,到时候再做引用。
\end{exercise}
So far, we have defined \engordnumber{1} order derivatives (Lie derivative,
covariant derivative) on functions and vector fields. Next we shall consider
\engordnumber{2} order derivatives of a function and vector fields.
In particular, non-commutative nature of \engordnumber{2} order derivative
on vector fields will be captured by the ``curvature''.
    
Let's assume \(\nabla\) to be the Levi-Civita connection from now on.
\begin{itemize}
    \item Recall we have defined tangent bundle. 
    \(TM=\bigcup_{p\in M} T_p M\) is the collection of all 
    tangent vector spaces. As a vector space, \(T_p M\) is isomorphic 
    to \(\mathbb{R}^n\). Let \(T^*_p M=\)\{all covectors at \(p\)\}.
    We also let \(T^*M=\bigcup_{p\in M}T^*_p M=\bigcup_p 
    \left\{(p,\alpha)| \alpha\in T^*_p M\right\}\). Similar to the 
    \(TM\), one can endow a smooth structure on \(T^*M\) such that 
    it's a smooth manifold of dimension \(2\dim M\). We shall also 
    call an element of \(\Gamma(T^* M)\) a differential 1-form, 
    where \(\Gamma(T^*M)\) is the collection of all global covectors.
    \item We have introduced 
    \[
        \nabla \colon \Gamma(TM)\times C^\infty(M)\to C^\infty(M)
    \]
    \[
        (X,f)\mapsto \nabla_X f.    
    \]
    This map induces 
    \[
        \nabla f\colon \Gamma(TM)\to C^\infty(M).
    \]
    At each point \(p\in M\) this map is 
    \[
        X\mapsto \nabla_X f=\nabla f(X).    
    \]
    \(
        \nabla f(p)\colon T_p M\to \mathbb{R}\Rightarrow 
        \nabla f(p)\in T^*_p M\Rightarrow
        \nabla f\in \Gamma(T^* M)
    \).
\end{itemize}
\begin{remark}
    If \((x^1,\ldots, x^n)\) is local coordinate at \(p\). \(T_p M
    =\Span\{\pdv{x^1},\ldots,\pdv{x^n}\}\), then \(T^*_p M=\Span 
    \{d x^1,\ldots , d x^n\}\), \(dx^i\left(\pdv{x^j}\right)=\delta^i_j\)
    \[
        \Rightarrow \nabla f(p)=\pdv{f}{x^i} (p)dx^i=\nabla_i f\cdot 
        d x^i. \quad (\nabla_i f=\pdv{f}{x^i})    
    \]
\end{remark}
\begin{definition}
    The gradient vector field of \(f\), written as \(\mathrm{grad} f=
    \mathrm{grad}_g f \) is defined as \(\forall X\in \Gamma(TM)\)
    \[
        g(\mathrm{grad} f,X)=\nabla_X f=\nabla f(X).    
    \]
\end{definition}
Locally, 
\[
    \mathrm{grad} f=g^{ij}\pdv{f}{x^j}\pdv{x^i}=\nabla^i f\pdv{x^i}.
\]
\begin{definition}[Hessian of \(f\)]
    \[
        \nabla\nabla f\colon \Gamma(TM)\times \Gamma(TM)\to C^\infty(M)
    \]
    \[
        (X,Y)\mapsto \nabla\nabla f(X,Y).    
    \]
    Where \(\nabla\nabla f(X,Y)=\left(\nabla_X \nabla f\right)(Y)
    \footnotemark
    =\nabla_X
    \left(\nabla f(Y)\right)-\nabla f\left(\nabla_X Y\right)=
    X\left(Y f\right)-()\nabla_X Y) f\).
\end{definition}
\footnotetext{Note that here we haven't defined the covariant derivative of
 a 1-form}
 Locally: 
 \begin{align*}
    \nabla\nabla f\left(\pdv{x^i},\pdv{x^j}\right)&=
    \pdv{x^i}\left(\pdv{f}{x^j}\right)-\nabla_{\pdv{x^i}}\pdv{x^j} f\\
    &=\pdv{f}{x^i}{x^j}-\Gamma\indices*{_{ij}^k}\pdv{f}{x^k}.
 \end{align*}
 \begin{remark}
    \begin{enumerate}[(1)]
        \item We conventionally write \(\nabla\nabla f(\pdv{x^i}
        \pdv{x^j})=\nabla_i\nabla_j f\), \ie\ 
        \[
            \nabla_i\nabla_j f=
            \pdv{f}{x^i}{x^j}-\Gamma\indices*{_{ij}^k}\pdv{f}{x^k}.   
        \]
        \begin{itemize}
            \item \(\nabla_i\nabla_j f\): tensor component.
            \item \(\pdv{f}{x^i}{x^j}\): Euclidean Hessian.
            \item \(\Gamma\indices*{_{ij}^k}\pdv{f}{x^k}\)
            Correction.
        \end{itemize}
        \item \(\nabla_i\nabla_j f=\nabla_j\nabla_i f\).(Symmetric in 
        \(i\) and \(j\))
        \item In tensor notation we write this as 
        \[
            \nabla\nabla f=\left(\nabla_i\nabla_j f\right)d x^i d x^j.    
        \]
    \end{enumerate}
 \end{remark}
 \begin{definition}[Laplacian of \(f\)]
    \(\Delta f=\mathrm{trace}\left(\nabla\left(\mathrm{grad}
    f\right)\right)\).
 \end{definition}
 Locally, \(\mathrm{grad} f=\nabla^i f\pdv{x^i}\Rightarrow
 \nabla\left(\mathrm{grad}f\right)=\nabla_j\nabla^i f dx^j\otimes
  \pdv{x^i}\).
  \begin{align*}
    \Delta f & =\nabla_i \nabla^i f=g^{ij}\nabla_i\nabla_j f\\
    &=g^{ij}\left(
        \pdv{f}{x^i}{x^j}-\Gamma\indices*{_{ij}^k}\pdv{f}{x^k}
    \right).
  \end{align*}
  \begin{exercise}
    Show that 
    \[
        \Delta f=\frac{1}{\sqrt{\det g}}\pdv{x^i}\left(
            \sqrt{\det g}\cdot g^{ij}\pdv{f}{x^j}
        \right)    .
    \]
  \end{exercise}
  Note that the expression of R.H.S follows from
  ``integration by parts'', \ie\ \(\forall h\in C^\infty_c(M)\),
  \[
    \int_M  h\Delta f \sqrt{g}\dd x =-\int_M \left\langle
        \nabla h,\nabla f
    \right\rangle\sqrt{g}\dd x.
  \]
  Next, we consider taking \engordnumber{2} order derivative
  of a vector field \(Z\) along two different vector fields \(X\) and 
  \(Y\). Let's do a general local computation.(The following 
  computation should be very familiar with you)

  Let \(X=X^i\pdv{x^i},Y=Y^j\pdv{x^j},Z=Z^k\pdv{x^k}\)
  \begin{align*}
    \Rightarrow \nabla_X\nabla_Y Z&=
    \nabla_X\left(\nabla_Y Z\right)\\
    &=\nabla_X\left(\left(Y^j \nabla_j Z^k\right)\pdv{x^k}\right)\\
    &=\left(X^i\nabla_i Y^j\nabla_j Z^k+X^i Y^j\nabla_j\nabla_i Z^k
    \right)\pdv{x^k}\\
    \nabla_Y\nabla_X Z&=\left(Y^j\nabla_j X^i \nabla_i Z^k
    +X^iY^j \nabla_j\nabla_i Z^k\right)\pdv{x^k}
  \end{align*}
  \begin{align*}
    &\Rightarrow \nabla_X\nabla_Y Z-\nabla_Y\nabla_X Z\\
    &=
    X^i Y^j\left(\nabla_i\nabla_j Z^k-\nabla_j\nabla_i Z^k\right)
    \pdv{x^k}+\left(\nabla_{\nabla_X Y}Z^k-\nabla_{\nabla
    _Y X }Z^k\right)\pdv{x^k}\\
    &=X^i Y^j\left(\nabla_i\nabla_j Z^k-\nabla_j\nabla_i Z^k\right)
    \pdv{x^k}
    +\left(\nabla_{[X,Y]}Z^k\right)\pdv{x^k}.
  \end{align*}
  \(\therefore \nabla_X\nabla_Y Z-\nabla_Y\nabla_X Z -\nabla_{[X,Y]}Z=
  X^i Y^j\left(\nabla_i\nabla_j Z^k-\nabla_j\nabla_i Z^k\right)
    \pdv{x^k}
  \).
\begin{definition}
    The (1,3) Riemannian curvature tensor is defined by a map 
    \[
        R\colon \Gamma(TM)\times \Gamma(TM)\times \Gamma(TM)
        \to \Gamma(TM)    
    \]
    \[
        \left(X,Y,Z\right)\mapsto R(X,Y)Z    
    \]
    \[
        R(X,Y)Z=\nabla_X\nabla_Y Z-\nabla_Y\nabla_X Z -\nabla_{[X,Y]}Z    
    \]
    Locally, 
    \[
        R\left(\pdv{x^i},\pdv{x^j}\right)\pdv{x^k}=R\indices{_{ij}^l_k}
        \pdv{x^l}    
    \]
\end{definition}
\begin{exercise}
    \begin{enumerate}[(1)]
        \item Check \(R\) is \(C^\infty\) in \(X,Y,Z\).
        \item Find a local expression for \(R\indices{_{ij}^l_k}\)
        in terms of Christoffel symbols.
    \end{enumerate}
\end{exercise}